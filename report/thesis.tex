% - For printing to paper I can use openright + monochrome:
% \documentclass[monochrome,final,a4paper,12 pt,twoside,openright]{report}
% - Draft: overfullrule is not needed with this since page breaks change because of microtype.
% \documentclass[draft,a4paper,12 pt,twoside]{report}\overfullrule=0cm
% - Final: to produce PDF and correct overfull boxes.
\documentclass[final,a4paper,12 pt,twoside]{report}

\title{Title to be discussed} % TODO: Αυτόματη δημιουργία κώδικα ρομποτικής εφαρμογής από λεκτική περιγραφή
\author{Ορέστης Φλώρος-Μαλιβίτσης}
% \date{} % TODO

% XXX: Delete on final document
% \overfullrule=0.3cm % allows to find overfull hboxes much quicker

% \setlength{\parskip}{1.3em}
\usepackage{fullpage}

% Argument 'table' for '\cellcolor{}' command.
% Should be high because it might get loaded by another package with different arguments:
% - http://tex.stackexchange.com/questions/87197/latex-error-option-clash-for-package-xcolor-for-table
% - http://tex.stackexchange.com/questions/83101/option-clash-for-package-xcolor
\usepackage[usenames,table]{xcolor}
% https://tex.stackexchange.com/a/401184
\usepackage[many]{tcolorbox}
\usepackage{adjustbox}

% --------------------- MATH ---------------------
% Math font options:
% \usepackage{euler}
\usepackage{libertinust1math}
% Math packages
\usepackage{amsmath}
\usepackage{array}
\usepackage{amsthm}
\usepackage{amssymb}
\usepackage{amsfonts}
\usepackage{mathtools}
\usepackage{mathspec}
\usepackage{fixmath}
%\usepackage{siunitx}

% ---------------------- FONTS & LANGUAGE ----------------------
% Polyglossia warning: due to a bug (https://github.com/reutenauer/polyglossia/issues/110) numbering with greek letters is incorrect.
% /usr/share/texmf-dist/tex/latex/polyglossia/gloss-greek.ldf must be edited.
% or \setmainlanguage[numerals=arabic]{greek} should be used instead.
\usepackage{polyglossia} % \textenglish
\usepackage{microtype}
\setmainlanguage{greek}
\setotherlanguages{english}
\defaultfontfeatures{Ligatures=TeX,Scale=MatchLowercase}
\usepackage[LGR,T1]{fontenc}
\newcommand{\anoteleia}{\raisebox{0.4ex}{\char"0387\relax}}

\setmainfont[Mapping=tex-text]{GFS Didot}
\setsansfont[Mapping=tex-text]{DejaVu Sans}
\setmonofont[Mapping=tex-text]{DejaVu Sans Mono}
% Other math fonts:
% \setmathsfont(Latin)[Scale=MatchLowercase]{Linux Libertine O}
% \setmathsfont(Digits,Greek)[Scale=MatchLowercase]{GFS Neohellenic Math}

\PolyglossiaSetup{greek}{indentfirst=true}
\PolyglossiaSetup{english}{indentfirst=true}

% ---------------------- BIBLIOGRAPHY ----------------------
% Add bibliography to toc table
\usepackage[notlot,nottoc,notlof]{tocbibind}

\usepackage[numbers,sort]{natbib}
\bibliographystyle{unsrtnat}

% ---------------------- OTHER ----------------------
% Extended graphics package
\usepackage{graphicx}
\graphicspath{{./images/}{./images/graphs/}}
\usepackage{tikz}
\usetikzlibrary{arrows.meta,fit,positioning}
\usepackage{pgf}
% \usetikzlibrary{external}\tikzexternalize % Uncomment to extract figures to separate PDFs
\usepackage{calc} % Used for \svgscale in inkscape

\setcounter{tocdepth}{2}

\usepackage[font=small]{caption}
% Sub-figures and sub-tables can be typeset using the subcaption or subfig package:
% \usepackage{subcaption}

\usepackage{paralist} % \compactitem, \compactenum

\usepackage{fancyhdr}
\setlength{\headheight}{15.2pt}
\pagestyle{fancy}
\renewcommand{\footrulewidth}{0.4pt}
\renewcommand{\headsep}{15pt}
\makeatletter\fancyfoot[CO,CE]{\small{\@title}}\makeatother
\fancyfoot[RO,LE]{\thepage}

\usepackage[avantgarde]{quotchap}
\usepackage[calcwidth]{titlesec}

\titleformat{\section}[hang]{\scshape}
{\Large\thesection}{12pt}{\Large}[{\titlerule[0.5pt]}]

\titleformat{\part}[display]{\huge}{
    \MakeUppercase{\partname} \Huge{\thepart}%
}{80pt}{}


\lhead[\leftmark]{}
\rhead[]{\rightmark}

\usepackage{xifthen} % \ifthenelse, \isempty
\usepackage[section]{placeins} % \FloatBarrier
\usepackage{tabulary} % Use with LCRJ, makes the total table width equal to <length>
\usepackage[english=american]{csquotes} % \enquote, \foreignquote
\usepackage{imakeidx}
\makeindex[options=-c -s thesis.ist,columns=1]{}

% --------------------- CODE ---------------------
% Minted package for the listings.
\usepackage[newfloat=true]{minted}
\setminted{%
    autogobble=true,%
    breaklines=true,%
    frame=single,%
    framerule=2pt,%
    linenos=false%
}
\newmintinline[python]{python}{}

% Change the caption name for listing / code enviroment.
\captionsetup[listing]{name=Καταχώρηση κώδικα,skip=0pt,position=below}
% and list of listings.
\SetupFloatingEnvironment{listing}{listname=Κατάλογος αλγορίθμων και καταχωρήσεων κώδικα}
% Environment with custom caption.
\newenvironment{code}{\centering\captionsetup{type=listing}}{}

% ---------------------- OTHER ----------------------

% http://ftp.cc.uoc.gr/mirrors/CTAN/macros/latex/contrib/hyperref/doc/manual.html
% https://tex.stackexchange.com/questions/1863/which-packages-should-be-loaded-after-hyperref-instead-of-before
\usepackage{float}
\floatplacement{figure}{ht}
\floatplacement{listing}{ht}
% TODO: pdfsubject, pdfkeywords
\usepackage[xetex,colorlinks=true,pdfborder={0 0 0},urlcolor=cyan,linkcolor=blue,citecolor=red,breaklinks,backref=page,pdfusetitle,unicode]{hyperref}
\renewcommand*{\backref}[1]{}
\renewcommand*{\backrefalt}[4]{{\footnotesize [%
                \ifcase #1 Δεν υπάρχει αναφορά%
                \or Αναφορά στη σελίδα~#2%
                \else Αναφορές στις σελίδες~#2%
                \fi%
            ]}}

\usepackage[plain]{fancyref}
\newcommand*{\fancyrefsubseclabelprefix}{subsec}
\newcommand*{\fancyreflstlabelprefix}{lst}
\fancyrefaddcaptions{greek}{
    \renewcommand*{\frefseename}{βλέπε}
    \renewcommand*{\Frefseename}{Βλέπε}
    \renewcommand*{\frefeqname}{εξίσωση}
    \renewcommand*{\Frefeqname}{Εξίσωση}
    \renewcommand*{\frefsecname}{ενότητα}
    \renewcommand*{\Frefsecname}{Ενότητα}
    \providecommand*{\frefsubsecname}{υποενότητα}
    \providecommand*{\Frefsubsecname}{Υποενότητα}
    \providecommand*{\freflstname}{καταχώρηση}
    \providecommand*{\Freflstname}{Καταχώρηση}
}
\frefformat{plain}{\fancyrefsubseclabelprefix}{\frefsubsecname\fancyrefdefaultspacing#1}
\Frefformat{plain}{\fancyrefsubseclabelprefix}{\Frefsubsecname\fancyrefdefaultspacing#1}
\frefformat{plain}{\fancyreflstlabelprefix}{\freflstname\fancyrefdefaultspacing#1}
\Frefformat{plain}{\fancyreflstlabelprefix}{\Freflstname\fancyrefdefaultspacing#1}

% ---------------------- CUSTOM COMMANDS ----------------------
% Captions with list entries
\newcommand{\ccaption}[2]{\caption[#1]{#1. Από~\protect\cite{#2}.}}
\newcommand{\lcaption}[2]{\caption[#1]{#1. #2}}
% Write in English
\newcommand{\en}[1]{\textenglish{#1}}
% Like \enquote
\newcommand{\engquote}[1]{\en{\foreignquote{english}{#1}}}
% How to display new terms
\let\newtermformat\MakeLowercase% Convert text to lowercase
\newcommand{\newtermprint}[2][]{\emph{\newtermformat{#2}}\ifthenelse{\isempty{#1}}{}{ (\en{#1})}}
\newcommand{\newtermprintc}[2][]{\emph{#2}\ifthenelse{\isempty{#1}}{}{ (\en{#1})}}
% Define terms with their English translation and place them on the index
\makeatletter
\newcommand{\newterm@generic}[4][]{%
\begingroup%
\def\rr##1##2{##1}% Text inside \rr{} normal
\def\dd##1{##1}% Text inside \dd{} normal
\newtermprint[#3]{#2}%
\endgroup%
\begingroup%
\def\rr##1##2{##2}% Text inside \rr{} normal
\let\dd\@gobble% Text inside \dd{} deleted
% Create the index argument
\ifthenelse{\isempty{#3}}{%
    \def\indexArg{#2}%
}{%
    \def\indexArg{#2 - #3}%
}%
\ifthenelse{\isempty{#1}}{}{%
    \edef\indexArg{#1!\indexArg}%
}%
\ifthenelse{\isempty{#4}}{%
\index{\indexArg}%
}{%
\index{\indexArg|see {#4}}%
}%
\endgroup%
}
\newcommand{\newterm}[3][]{\newterm@generic[#1]{#2}{#3}{}}
\newcommand{\newtermc}[3][]{\begingroup\let\newtermformat\@firstofone\newterm@generic[#1]{#2}{#3}{}\endgroup}
\newcommand{\newtermsee}[4][]{\newterm@generic[#1]{#2}{#3}{#4}}
\newcommand{\newtermcsee}[4][]{\begingroup\let\newtermformat\@firstofone\newterm@generic[#1]{#2}{#3}{#4}\endgroup}
% Figure shortcut
\newlength{\ig@indent}
\setlength{\ig@indent}{\parindent}
\define@key{ig}{pos}{\def\ig@pos{#1}} % Position of figure like htbpH
\define@key{ig}{inc}{\def\ig@inc{#1}} % \includegraphics options like width=
\define@key{ig}{type}{\def\ig@type{#1}} % \includegraphics options like width=
\define@key{ig}{svgwidth}{\ifthenelse{\isempty{#1}}{}{\def\svgwidth{#1}}} %
\define@key{ig}{svgscale}{\ifthenelse{\isempty{#1}}{}{\def\svgscale{#1}}} %
\newcommand{\ig@maxfigure}[3]{%
    \bgroup%
    \sbox0{\begin{minipage}{\textwidth}#3\end{minipage}}%
    \ifdim\ht0>#1\relax\errmessage{\string\maxfigure: caption too big}%
    \else\centering\maxsizebox{\textwidth}{\dimexpr #1 - \ht0 - \abovecaptionskip}{#2}\fi%
    \par\box0%
    \egroup%
}
% https://tex.stackexchange.com/a/20892/
\newcommand{\ig@escapeunderscore}{\begingroup\@makeother\_\@escapeunderscore}
\newcommand*{\@escapeunderscore}[1]{#1\endgroup}
\setkeys{ig}{inc=,pos=ht,type=pgf,svgwidth=,svgscale=}% The defaults
\newcommand{\ig}[3][]{% #1 = the keys, #2 = file name == label #3 = \caption-like command
    \begingroup% Otherwise \setkeys will affect subsequent calls
    \setkeys{ig}{#1}% Set new keys
    \expandafter\figure\expandafter[\ig@pos]% https://tex.stackexchange.com/a/11340/
    \centering% Just sets \par parameters, can be used everywhere
    \captionsetup{singlelinecheck=false}%
    % \captionsetup{singlelinecheck=false,format=hang,hangindent=-\ig@indent}% TODO: make sure this looks nice for small captions as well
    \ifthenelse{\equal{\ig@type}{pgf} \OR \equal{\ig@type}{tikz}}{%
        \input{images/#2.\ig@type}% Not affected by \graphicspath
        #3\label{fig:#2}
    }{%
        \ifthenelse{\equal{\ig@type}{inkscape}}{%
            \ig@maxfigure{%
                0.95\textheight - \topskip - \baselineskip}{%
                \ig@escapeunderscore{\input{images/graphs/#2.pdf_tex}}}{%
                #3\label{fig:#2}}%
        }{%
            \expandafter\includegraphics\expandafter[\ig@inc]{#2.\ig@type}% Affected by \graphicspath
            #3\label{fig:#2}
        }%
    }%
    \endfigure%
    \endgroup%
}
\makeatother
% My math stuff
\DeclarePairedDelimiter{\abs}{\lvert}{\rvert}
\renewcommand{\vec}[1]{\mathbold{#1}}
\newcommand{\mat}[1]{\mathbold{#1}}
\newcommand{\vx}{\vec{x}}
\newcommand{\vX}{\mat{X}}
\newcommand{\vy}{\vec{y}}
\newcommand{\vY}{\mat{Y}}
\newcommand{\vw}{\vec{w}}
\newcommand{\vth}{\vec{\theta}}
\newcommand{\vTh}{\mat{\Theta}}
\newcommand{\hth}{h_{\vth}}
\newcommand{\hTh}{h_{\vTh}}
\newenvironment{conditions}{% "Όπου:" …
\\\vspace{\abovedisplayskip}\noindent\tabulary{\linewidth}{>{$}L<{$} @{${}\hspace{1em}{}$} L}
}{%
\endtabulary\par\vspace{\belowdisplayskip}%
}
% For intent formatting
\newcommand{\intent}[1]{\mintinline{text}{#1}}
\newcommand{\entity}[1]{\mintinline{text}{#1}}
\newcommand{\intentr}[1]{%
    \begingroup%
    \def\tmp{#1}%
    \StrDel{\tmp}{ }[\tmp]%
    \hypersetup{linkcolor=black}%
    \hyperref[intent:\tmp]{\expandafter\intent\expandafter{\tmp}}%
    \endgroup%
}
\newcommand{\noslot}{\ensuremath{\varnothing}}
% Formatting shortcuts
\newcommand{\bfttt}[1]{\textbf{\texttt{#1}}}
\newcommand{\enttt}[1]{\en{\texttt{#1}}}
\newcommand{\enbfttt}[1]{\en{\bfttt{#1}}}
% Cite + format a library
\newcommand{\lib}[1]{\en{#1}}
\newcommand{\libcite}[2][]{%
    \lowercase{\def\tmp{#2}}%
    \lib{#2}\index{#1#2}~\cite{\tmp}%
}
% Break a url if needed
\newcommand{\breakurl}[1]{%
    \begingroup%
    \def\UrlBigBreaks{\do\/\do-\do:}%https://tex.stackexchange.com/a/269122/
    \url{#1}%
    \endgroup%
}
% hyperref shortcuts
\newcommand{\TOCr}{\hyperref[sec:TOC]{\MakeLowercase{\contentsname}}}
\newcommand{\TOCR}{\hyperref[sec:TOC]{\contentsname}}
\newcommand{\SRLR}{\hyperref[subsec:srl]{\en{SRL}}}
\newcommand{\CRFR}{\hyperref[sec:crf]{\en{CRF}}}
\newcommand{\NLU}{\hyperref[subsec:4-NLU]{\en{NLU}}}
% project names
\newcommand{\metamodel}{\en{R4A-NAO}}
\newcommand{\projectname}{\en{r4a-nao-nlp}}
% SRL arguments
\newcommand{\ARGs}{\enttt{ARG0}\ldots{}\enttt{ARG5}}
\newcommand{\ARGM}{\enttt{ARGM}}

% ---------------------- MAIN DOCUMENT ----------------------
\begin{document}
% Hyphenation rules after document start
% https://tex.stackexchange.com/questions/229915/polyglossia-ignores-my-custom-hyphenation
\begin{english}\hyphenation{neu-ral-co-ref Co-ref-An-no-ta-tor Al-len-NLP Core-NLP Rust-ling mat-plot-lib}\end{english}

% https://tex.stackexchange.com/questions/4705/
% https://english.stackexchange.com/questions/2544/
% Causes LaTeX to treat spacing between sentences in the same way as spacing between words in the middle of a sentence.
\frenchspacing
% Causes spacing between sentences to stretch or shrink more (see \spacefactor); this is the default.
% \nonfrenchspacing

% TODO: \newevenside?
\pagenumbering{roman}
\begin{titlepage}
    % AUTH Logo
    \centering\begin{minipage}{0.3\textwidth}
        \centering\includegraphics[height=3cm]{university.png}
    \end{minipage}%
    \begin{minipage}{0.7\textwidth}
        \begin{flushleft}
            \large Αριστοτέλειο Πανεπιστήμιο Θεσσαλονίκης\\
            Πολυτεχνική Σχολή\\
            Τμήμα Ηλεκτρολόγων Μηχανικών \&\\Μηχανικών Υπολογιστών\\
            Τομέας Ηλεκτρονικής και Υπολογιστών\\
            Ομάδα Ευφυών Συστημάτων και Τεχνολογίας Λογισμικού (ISSEL)\\[5cm]
        \end{flushleft}
    \end{minipage} \\[1.7cm]

    \begin{center}
        \Large Διπλωματική Εργασία \\[0.8cm]

        \rule{450pt}{4pt} \\[0.4cm]
        \makeatletter{\fontsize{20.26pt}{1em}\selectfont{}\@title{}}\makeatother

        \rule{350pt}{4pt} \\[4cm]

        \noindent\begin{minipage}{0.5\textwidth}
            \begin{flushleft} \large
                \emph{Εκπόνηση:} \\
                Φλώρος-Μαλιβίτσης Ορέστης\\ΑΕΜ:7796\\
            \end{flushleft}
        \end{minipage}\hfill%
        \begin{minipage}{0.5\textwidth}
            \begin{flushright} \large
                \emph{Επίβλεψη:} \\
                Ανδρέας Λ. Συμεωνίδης --\\ Αναπληρωτής καθηγητής\\
                Εμμανουήλ Τσαρδούλιας --\\ Μεταδιδάκτορας
            \end{flushright}
        \end{minipage}
        \\[1cm]
        \vfill

        \large Θεσσαλονίκη, \today{} % TODO
    \end{center}
\end{titlepage}

% vim:ts=4:sw=4:expandtab:fo-=tc:tw=120


\begin{center}
    \textbf{\Large{Ευχαριστίες}}\phantomsection
    \addcontentsline{toc}{section}{Ευχαριστίες}
\end{center}

Θα ήθελα να ευχαριστήσω θερμά τον Αναπληρωτή Kαθηγητή κ. Ανδρέα Συμεωνίδη για την εμπιστοσύνη που μου έδειξε με την ανάθεση της εργασίας
και τον Δρ. Εμμανουήλ Τσαρδούλια για την υπομονή του, τη συνεχή στήριξη και καθοδήγηση που μου προσέφερε όποτε αυτή χρειάστηκε κατά τη διάρκεια της εκπόνησης της παρούσας διπλωματικής.
Επίσης, ευχαριστώ τον Δρ. Χριστόφορο Ζολώτα για το χρόνο του και τις επεξηγήσεις του πάνω στο μέτα-μοντέλο \metamodel{}.

Επιπρόσθετα, θέλω να ευχαριστήσω την οικογένεια μου που στάθηκε δίπλα μου σε όλη τη διάρκεια των σπουδών μου και ιδιαίτερα τους γονείς μου για τη συμπαράστασή τους στη συγγραφή αυτής της εργασίας.

Τέλος, θέλω να ευχαριστήσω όλους τους φίλους που με συντρόφευσαν στα φοιτητικά μου χρόνια και με στήριξαν στην εκπόνηση της παρούσας.

% vim:ts=4:sw=4:expandtab:fo-=tc:tw=120
      \clearpage
\begin{center}
    \centering

    \vspace{0.5cm}
    \centering
    \textbf{\Large{Περίληψη}}
    \phantomsection
    \addcontentsline{toc}{section}{Περίληψη}

    \vspace{1cm}
\end{center}

TODO

% vim:ts=4:sw=4:expandtab:fo-=tc:tw=120
 \clearpage
\begin{english}
\phantomsection
\addcontentsline{toc}{section}{Abstract}

\begin{center}
    \centering
    \textbf{\Large{Diploma Thesis}}

    \vspace{0.5cm}
    \large{English title to be discussed}

    \vspace{1cm}
    \centering
    \textbf{Abstract}
\end{center}

% TODO

\begin{flushright}
    \vspace{2cm}
    Floros-Malivitsis Orestis\\
    Electrical \& Computer Engineering Department,\\
    Aristotle University of Thessaloniki, Greece\\
    June 2019
\end{flushright}
\end{english}

% vim:ts=4:sw=4:expandtab:fo-=tc:tw=120
 \clearpage

\tableofcontents\label{sec:TOC}
\listoffigures
\listoflistings

\chapter*{Ακρωνύμια Εγγράφου}\phantomsection
\addcontentsline{toc}{section}{Ακρωνύμια}

% vim:ts=4:sw=4:expandtab:fo-=tc:tw=120
 % TODO: same page with listoflistings?

% vim:ts=4:sw=4:expandtab:fo-=tc:tw=120

\clearpage{}
\pagenumbering{arabic}
\chapter{Εισαγωγή}\label{chap:intro}

\section{Κίνητρο}\label{sec:motivation}
Από τα αρχαία χρόνια, ο άνθρωπος ονειρευόταν μηχανές που μιμούνται οργανισμούς ή ακόμα και ξεπερνούν τους ανθρώπους στις ικανότητές τους.
Μια από τις αρχαιότερες αναφορές σε αυτόνομα, αυτοκινούμενα ρομπότ πρέπει να είναι αυτή στην Ιλιάδα του Όμηρου, γύρω στο 800 π.Χ.
Εκεί, αναφέρεται πως ο Ήφαιστος κατασκεύασε τρίποδες \enquote{με ρόδες χρυσές για να μπορούν αυτόματα να μπαίνουν στων θεών τη σύναξη και πάλι μόνοι τους να γυρνούν στο οίκημα}.
Οι τρίποδες αυτοί περιγράφονται ως ισχυροί και έξυπνοι, με αυτιά και φωνές, πρόθυμοι να βοηθήσουν και να εργαστούν~\cite{graefe2009ancient}.

Η λέξη \emph{ρομπότ} προέρχεται από την τσεχική λέξη \enquote{robota} που σημαίνει δουλειά, εργασία.
Επινοήθηκε από τον αδερφό του Τσέχου θεατρικού συγγραφέα Κάρελ Τσάπεκ (Karel Čapek) το 1920.
Η λέξη έχει καταλήξει να περιλαμβάνει κάθε μηχανική συσκευή που μπορεί να εκτελέσει εργασίες οι οποίες κατά κύριο λόγο εκτελούνται από ανθρώπους.
Συνήθως, θεωρείται ότι η συσκευή αυτή έχει ανθρωποειδή χαρακτηριστικά, αν και αυτό δεν είναι αναγκαστικό~\cite{asimov1989asimov}.

Ο κλάδος της ρομποτικής ασχολείται με τη σύνθεση διάφορων ανθρώπινων λειτουργιών με τη χρήση ποικίλων μηχανισμών, αισθητήρων, ενεργοποιητών και υπολογιστών.
Η σύνθεση αυτή αποτελεί τεράστιο εγχείρημα και απαιτεί τον δανεισμό πληθώρας ιδεών από πολλούς \enquote{κλασικούς} επιστημονικούς κλάδους~\cite{craig2009introduction}.

Καθώς η τεχνολογία που χρησιμοποιούμε καθημερινά γίνεται ολοένα και πιο περίπλοκη, προσπαθούμε να βρίσκουμε τρόπους που καθιστούν οικεία και φιλική την αλληλεπίδραση με αυτή.
Οι άνθρωποι και άλλα πρωτεύοντα θηλαστικά έχουν εξελιχθεί ώστε να έχουν εγκέφαλους ικανούς για άριστη κοινωνική αλληλεπίδραση,
οπότε η προσπάθεια χρήσης της φυσικής γλώσσας για την επικοινωνία με μηχανήματα είναι ένας λογικά επόμενος στόχος~\cite{dunbar2007evolution,breazeal2004designing}.
Ακόμα και στην αρχαία μυθολογία, ο Ήφαιστος έδωσε ανθρώπινη φωνή στις χρυσές μηχανικές υπηρέτριές του, καθιστώντας τες πιο αποτελεσματικά εργαλεία~\cite{gera2003ancient}.

Τα \enquote{παραδοσιακά} συστήματα ρομπότ συνήθως σχεδιάζονται από επαγγελματίες μηχανικούς και ο προγραμματισμός τους γίνεται επίσης από ειδικούς και αφορά σε ένα συγκεκριμένο σκοπό.
Για παράδειγμα, ένα βιομηχανικό ρομπότ απαιτεί την εργασία ηλεκτρολόγων και μηχανολόγων μηχανικών καθώς και διάφορων μηχανικών λογισμικού για να επιτευχθεί η άρτια συνεργασία των αισθητήρων, των μηχανολογικών μερών και του λογισμικού.
Για αυτό τον λόγο, η δυνατότητα προγραμματισμού, σε υψηλό τουλάχιστον επίπεδο, ρομποτικών εφαρμογών μπορεί να βοηθήσει στη σημαντική μείωση του κόστους ανάπτυξης και να επιτρέψει σε άτομα χωρίς επαγγελματική εμπειρία να χρησιμοποιήσουν προγραμματιζόμενα ρομποτικά συστήματα γενικής χρήσης όπως το NAO\footnote{\url{https://www.softbankrobotics.com/emea/en/nao}}.

\section{Περιγραφή του προβλήματος}\label{sec:problem-description}
Το γενικό πρόβλημα που τίθεται προς λύση είναι η αυτόματην παραγωγή μιας ρομποτικής εφαρμογής με βάση την περιγραφή της από τον χρήστη σε φυσική γλώσσα.
Ένα κατάλληλα εξοπλισμένο ρομπότ θα μπορεί να εκτελέσει τον παραγόμενο κώδικα, μεταφέροντας τη διαδικασία που περιγράφεται στον φυσικό κόσμο.

Πιο συγκεκριμένα, γίνεται προσπάθεια αντιστοίχησης ενός κειμένου αγγλικής γλώσσας σε ένα σύνολο προκαθορισμένων ενεργειών που προσφέρονται από μια ανεξάρτητη πλατφόρμα λογισμικού.
Η παρούσα διπλωματική ασχολείται με το κομμάτι της στατικής αντιστοίχησης των προτάσεων του κειμένου στις κατάλληλες ενέργειες.
Δεν γίνεται προσπάθεια συνδυασμού τους για την παραγωγή ενός ολοκληρωμένου αλγορίθμου που υλοποιεί τη λογική που προδιαγράφεται στο κείμενο,
το πρόβλημα αυτό θεωρείται εκτός του πεδίου μελέτης και μπορεί να επιλυθεί ξεχωριστά αξιοποιώντας τα αποτελέσματα αυτής της εργασίας.

\section{Στόχοι της διπλωματικής}\label{sec:diploma-purpose}
Στόχος της συγκεκριμένης διπλωματικής εργασίας είναι η επεξεργασία γραπτών προτάσεων μέσω τεχνικών
\newterm{Επεξεργασία\dd{ς} Φυσικής Γλώσσας}{Natural Language Processing - NLP}.
Οι προτάσεις αυτές περιγράφουν μια ρομποτική διαδικασία που μπορεί να εκτελεστεί από το δοθέν ρομπότ.

Η ρομποτική αυτή διαδικασία αποτελείται από προκαθορισμένες ενέργειες που περιέχονται στο μετα-μοντέλο \metamodel{}\footnote{\url{https://r4a.issel.ee.auth.gr/nao4a/}}.
Σε κάθε πρόταση γίνεται προσπάθεια αναγνώρισης των \newterm{\rr{προθέσεων}{Αναγνώριση Πρόθεσης}}{Intent\rr{}{ Identification}} του χρήστη,
που θεωρείται ότι αντιστοιχούν σε κάποιες από τις προκαθορισμένες ενέργειες του μέτα-μοντέλου.

Δεν γίνεται προσπάθεια επίλυσης ορθογραφικών, συντακτικών και γραμματικών λαθών στο κείμενο εισόδου.

\section{Διάρθρωση εγγράφου}\label{sec:structure}
Η αναφορά της παρούσας διπλωματικής εργασίας περιλαμβάνει έξι κεφάλαια:
\begin{compactenum}
    \item Το \hyperref[chap:intro]{τρέχον} αποτελεί την εισαγωγή.
    \item Το \hyperref[chap:background]{δεύτερο} παρουσιάζει το σχετικό υπόβαθρο\anoteleia{}
          βασικές έννοιες που χρησιμοποιήθηκαν καθώς και τους αλγορίθμους μηχανικής μάθησης στους οποίους βασίζεται η υλοποίηση της διπλωματικής.
    \item Το \hyperref[chap:state-of-the-art]{τρίτο} συνοψίζει την ερευνητική περιοχή σχετική με το αντικείμενο της εργασίας.
    \item Το \hyperref[chap:methodology]{τέταρτο} αποτελείται από την επεξήγηση της μεθοδολογίας που ακολουθήθηκε και την λειτουργία της σωλήνωσης λογισμικού που υλοποιήθηκε.
    \item Το \hyperref[chap:results]{πέμπτο} παρουσιάζει τα αποτελέσματα του \projectname{} σε διάφορα σενάρια χρήσης.
    \item Το \hyperref[chap:conclusions_future]{έκτο} καταλήγει με τα συμπεράσματα και τις πιθανές μελλοντικές επεκτάσεις.
\end{compactenum}

% vim:ts=4:sw=4:expandtab:fo-=tc:tw=120

\chapter{Υπόβαθρο}\label{chap:background}

% vim:ts=4:sw=4:expandtab:fo-=tc:tw=120

\chapter{Επισκόπηση Βιβλιογραφίας}\label{chap:state-of-the-art}
Σε αυτό το κεφάλαιο παρουσιάζονται αποσπάσματα της σχετικής βιβλιογραφίας που μελετήθηκε
και σύγχρονες μέθοδοι σχετικές με τους στόχους αυτής της εργασίας και παρόμοιες με τη μεθοδολογία που τελικώς ακολουθήθηκε.

\section{Συστήματα διαλόγου}
Ως \newterm{Σύστημα Διαλόγου}{Dialogue System}
ή \newtermsee{Διαδραστικό Πράκτορα Ομιλίας}{Interactive Conversational Agent}{Σύστημα Διαλόγου - Dialogue System}
ή \en{chatterbot / chatbot}
ορίζουμε ένα σύστημα σχεδιασμένο να αλληλεπιδρά με έναν χρήστη χρησιμοποιώντας φυσική γλώσσα.

\begin{figure}
    \centering
    \includegraphics{dialogue-system-diagram}
    \ccaption{Ένα σύστημα διαλόγου}{serban2015survey}
    \label{fig:dialogue-system-diagram}
\end{figure}

Σε αντίθεση με τους στόχους αυτής της διπλωματικής, τα συστήματα διαλόγου χαρακτηρίζονται από μεγαλύτερη αλληλεπίδραση με τον χρήστη.
Τους δίνεται η δυνατότητα να θέτουν ερωτήσεις για την επίλυση πιθανών απροσδιοριστιών και έχουν τη δυνατότητα να δίνουν απαντήσεις σε ερωτήματα του χρήστη.
Ωστόσο, μερικές βασικές αρχές και μέθοδοι που εφαρμόζονται στη βιβλιογραφία μπορούν να χρησιμοποιηθούν και στο πλαίσιο αυτής της εργασίας.

Ένας πιθανός διαχωρισμός των διάφορων συστημάτων διαλόγου γίνεται σύμφωνα με τον σκοπό χρήσης τους:
στα \newtermprint[Goal-driven Dialogue Systems]{Συστήματα Διαλόγου Προσανατολισμένα για Καθήκοντα}
και στα \newtermprint[Non-goal-driven Dialogue Systems]{Συστήματα Διαλόγου που δεν είναι Προσανατολισμένα για Καθήκοντα}.
Ωστόσο, ο διαχωρισμός αυτός δεν είναι πάντα αυστηρός και μερικά συστήματα μπορεί να συνδυάζουν χαρακτηριστικά και από τις δύο κατηγορίες~\cite{wang2016recent}.

\subsection{Συστήματα διαλόγου που δεν είναι προσανατολισμένα για καθήκοντα}\label{subsec:non-goal-driven-dialogue-systems}%
\index{Σύστημα Διαλόγου - Dialogue System!Μη Προσανατολισμένα για Καθήκοντα - Non-goal-driven}
% TODO:sec? subsec?
Αποτελούν μοντέλα που δεν έχουν έναν συγκεκριμένο σκοπό και προσπαθούν να διεξάγουν διάλογο προσομοιώνοντας έναν ανθρώπινο συνομιλητή.
Παραδείγματα χρήσεων αποτελούν η εκμάθηση γλώσσας ή απλώς η διασκέδαση~\cite{shawar2007chatbots,serban2015hierarchical}.

Ένα από τα πρώτα συστήματα διαλόγου ήταν η ELIZA~\cite{weizenbaum1966eliza} που προσπαθούσε να προσομοιώσει τις απαντήσεις ενός ψυχοθεραπευτή σε μια ψυχιατρική συνέντευξη.
Η ELIZA όμως δεν είχε κάποιον τρόπο να ακολουθεί τη συζήτηση ή να απαντάει σύμφωνα με το παρελθόν.
Έχει θεωρηθεί~\cite{shieber1994lessons} ότι η αρχική επιτυχία που είχε η ELIZA μπορεί να αποδοθεί στο ότι οι άνθρωποι μπορούν εύκολα να ξεγελαστούν αποδίδοντας δομή στο χάος, δίνοντας σημασία εκεί που δεν υπάρχει.
Στην ίδια κατεύθυνση, ένα άλλο σύστημα διαλόγου, ο PARRY~\cite{colby1981modeling}, προσπαθούσε να μιμηθεί τη συμπεριφορά ενός παρανοϊκού ασθενούς.
Το σύστημα ήταν ανώτερο καθώς είχε επίγνωση του ιστορικού της συζήτησης και της κατάστασης του μυαλού του.
Ωστόσο, όπως και η ELIZA, ο PARRY δεν έβγαζε συμπεράσματα ούτε \enquote{σκεφτόταν} την απάντησή του αλλά αναγνώριζε κάποια πρότυπα στην είσοδό του~\cite{colby1974ten}.
Κανένα από αυτά τα δύο συστήματα δεν χρησιμοποιούσε προσεγγίσεις μηχανικής μάθησης.

Αργότερα~\cite{shawar2007chatbots,serban2015survey}, εμφανίστηκαν συστήματα με διαφορετικές αρχιτεκτονικές.
Ένα από τα πρώτα που χρησιμοποιούσε μηχανική εκμάθηση είναι το \lib{MegaHal}~\cite{hutchens1998introducing} που αναπτύχθηκε για την εισαγωγή του στον διαγωνισμό \lib{Loebner}.
Μοντελοποιούσε τον διάλογο ως στοχαστική ακολουθία διακριτών συμβόλων με Μαρκοβιανές αλυσίδες.
Οι πιθανές απαντήσεις του \lib{MegaHal} βασίζονταν στην επιλογή λέξεων-κλειδιών από την είσοδο του χρήστη.
Αφού παρήγαγε αρκετές απαντήσεις, επέλεγε αυτή που προσέφερε την περισσότερη πληροφορία -- μεγαλύτερη εντροπία.
Το σύστημα άρχιζε με μηδενική γνώση της γλώσσας και εκπαιδευόταν σε έναν συνδυασμό πηγών:
\begin{compactitem}
    \item Προτάσεις που περιείχαν ένα φτιαχτό όνομα, ηλικία, απασχόληση και προσπαθούσαν να δώσουν \enquote{χαρακτήρα} στον \lib{MegaHal}
    \item Εγκυκλοπαιδικές γνώσεις
    \item Διάλογοι από ταινίες και σειρές
    \item Γνωστά αποσπάσματα
    \item Μερικά κείμενα σε γλώσσες διάφορες της αγγλικής
\end{compactitem}

Στην ίδια περίοδο, οι~\citet{levin1997stochastic} και \citet{levin1998using} πρότειναν για πρώτη φορά τη χρήση \newterm{Διαδικασ\rr{ιών}{ίες} Απόφασης Μάρκοβ}{Markov Decision Process - MDP}
και αργότερα αναπτύχθηκαν αρκετά συστήματα βασισμένα σε αυτές.
Για παράδειγμα, το RLDS~\cite{singh2000reinforcement}, ένα γενικό εργαλείο για συστήματα διαλόγου,
και το ELVIS~\cite{walker1998learning}, μια \newterm{Διεπαφή Ομιλούμενης Γλώσσας}{Spoken Language Interface - SLI} για την πρόσβαση σε μηνύματα ηλεκτρονικού ταχυδρομείου μέσω τηλεφώνου.
Στη συνέχεια, έγινε ιδιαίτερα διαδεδομένη~\cite{young2013pomdp,wang2016recent,roy2000spoken,young2002talking} η χρήση
\newterm[Διαδικασίες Απόφασης Μάρκοβ - Markov Decision Process - MDP]{Μερικώς Παρατηρήσιμ\rr{ων}{ες} \dd{Διαδικασιών Απόφασης Μάρκοβ}}{Partially Observable \dd{Markov Decision Process} - POMDP} %TODO:lowercase
στην οποία θεωρείται ότι ο διάλογος εξελίσσεται ως μια διαδικασία Μάρκοβ: ξεκινώντας από κάποια αρχική κατάσταση $s_0$, κάθε επόμενη κατάσταση διαμορφώνεται από μια πιθανότητα μετάβασης.
Το μοντέλο εκπαιδεύεται στα δεδομένα για να μάθει την καλύτερη πολιτική διαλόγου.

% TODO: hyperref?
Πιο σύγχρονα συστήματα διαλόγου εφαρμόζουν μοντέλα που βασίζονται σε αρχιτεκτονικές που αξιοποιούν νευρωνικά δίκτυα αλλά αυτά απαιτούν τη χρήση μεγάλου πλήθους δεδομένων για την εκπαίδευσή τους~\cite{serban2015survey}.
Σε αυτά περιλαμβάνονται τα \newtermprint[end-to-end]{άκρη-προς-άκρη} μοντέλα~\cite{serban2015survey,serban2016building}
που αντί να υλοποιούν ένα σύστημα πολλών τμημάτων με ξεχωριστές λειτουργίες,
εκπαιδεύονται ως ένα ενιαίο μοντέλο - \enquote{μαύρο κουτί} - στις εισόδους και τις εξόδους του επιθυμητού συστήματος διαλόγου.

\subsection{Συστήματα διαλόγου προσανατολισμένα για καθήκοντα}\label{subsec:goal-driven-dialogue-systems}%
\index{Σύστημα Διαλόγου - Dialogue System!Προσανατολισμένα για Καθήκοντα - Goal-driven}
% TODO:sec? subsec?
Αυτά τα μοντέλα συνήθως αξιοποιούν τεχνικές για τη μοντελοποίηση του διαλόγου παρόμοιες με αυτές των συστημάτων διαλόγου που δεν είναι προσανατολισμένα για καθήκοντα αλλά,
επιπλέον, περιλαμβάνουν τη χρήση τεχνικών που σχετίζονται με την εκτέλεση κάποιας πράξης σύμφωνα με τις επιθυμίες του χρήστη.

Μια βασική μονάδα που συναντάται σε αυτά τα συστήματα και μας ενδιαφέρει για την υλοποίηση αυτής της διπλωματικής είναι αυτό της
\newterm{Κατανόηση\dd{ς} Φυσικής Γλώσσας}{Natural Language Understanding - NLU}.
Στόχο της αποτελεί η αντιστοίχηση του κειμένου σε νόημα, δηλαδή η παραγωγή μιας σημασιολογικής αναπαράστασης ενός κειμένου~\cite{martin2009speech}.

\subsubsection{Προθέσεις και οντότητες}\label{subsec:intents-and-entities}
\begin{figure}
    \centering
    \begin{tabular}{|lcccccccc|}
        \hline
        \textbf{Λέξεις}       & book                                       & flight       & from         & Thessaloniki & to           & Reykjavík    & this         & weekend      \\
                              & $\downarrow$                               & $\downarrow$ & $\downarrow$ & $\downarrow$ & $\downarrow$ & $\downarrow$ & $\downarrow$ & $\downarrow$ \\
        \textbf{Slots}        & \noslot{}                                  & \noslot{}    & \noslot{}    & B-departure  & \noslot{}    & B-arrival    & B-date       & I-date       \\
        \textbf{Entity Types} & \noslot{}                                  & \noslot{}    & \noslot{}    & B-city       & \noslot{}    & B-city       & B-date       & I-date       \\
        \textbf{Intent}       & \multicolumn{8}{l|} {\intent{book_flight}}                                                                                                          \\
        \hline
    \end{tabular}
    \lcaption{Παράδειγμα αναγνώρισης πρόθεσης και εξαγωγής οντοτήτων}{%
        Η επισήμανση γίνεται με τη μορφή \newterm{Εσωτερικό-Εξωτερικό-Έναρξη}{Inside–Outside–Beginning - IOB}.
        Οι οντότητες \engquote{Thessaloniki} και \engquote{Reykjavík} έχουν τον ίδιο τύπο, αυτό της πόλης,
        αλλά πληρούν διαφορετικά σημασιολογικά ορίσματα στην πρόθεση: η πρώτη αποτελεί την τοποθεσία αναχώρησης και η άλλη την τοποθεσία άφιξης.%
    }%
    \label{fig:ex-intent-entity}
\end{figure}
Η πρώτη βασική διεργασία της NLU στα συστήματα διαλόγου προσανατολισμένα για καθήκοντα είναι
η \newterm{Αναγνώριση Πρόθεσης}{Intent Identification} ή \newtermsee{Ταξινόμηση Πρόθεσης}{Intent Classification}{Αναγνώριση Πρόθεσης - Intent Identification}
του χρήστη.
Μια πρόθεση είναι ένας σκοπός ή ένας στόχος που υποδηλώνεται από τα λεγόμενα του χρήστη,
αποτελεί μια παράγωγη σημασιολογική αναπαράσταση των λεγόμενων του χρήστη.
Για παράδειγμα, στην πρόταση \enquote{What's the weather like in Thessaloniki?} ο στόχος του χρήστη είναι να μάθει την κατάσταση του καιρού.
Στη διαδικασία της αναγνώρισης, το σύστημα ταξινομεί τα λεγόμενα του χρήστη σε μια ή περισσότερες προκαθορισμένες προθέσεις και εξάγει τις σχετικές οντότητες.
Συνήθως, οι προκαθορισμένες προθέσεις σχεδιάζονται από εμπειρογνώμονες σύμφωνα με τον τομέα της εφαρμογής
και τα συστήματα εκπαιδεύονται πάνω σε επισημασμένα σύνολα δεδομένων~\cite{tur2005semi}.

Συγγενικό στόχο με την αναγνώριση πρόθεσης αποτελεί η \newterm{Εξαγωγή Οντοτήτων}{Entity Extraction} και η \newterm{Πλήρωση Υποδοχέων}{Slot Filling}.
Αν και μερικές φορές οι δύο όροι χρησιμοποιούνται ως ισοδύναμοι,
οι υποδοχείς λειτουργούν ως σημασιολογικά ορίσματα που το μοντέλο καλείται να καλύψει ώστε να προσδιοριστούν οι λεπτομέρειες της πρόθεσης του χρήστη
ενώ οι οντότητες χαρακτηρίζονται από κλάσεις και παίρνουν τιμές άσχετα με την πρόθεση μέσα στην οποία εμφανίζονται~\cite{mesnil2013investigation,snips} (βλέπε και~\fref{fig:ex-intent-entity}).
Το ζητούμενο εδώ είναι η εκχώρηση μιας σημασιολογικής έννοιας σε κάθε λέξη της πρότασης~\cite{vu2016bi}.
Για παράδειγμα, στην πρόταση \engquote{What's the weather like in Thessaloniki?} η λέξη \engquote{Thessaloniki} πρέπει να επισημανθεί ως η πόλη για την οποία ο χρήστης θέλει να μάθει τον καιρό.
Οι οντότητες συνδέονται στενά με τις προκαθορισμένες προθέσεις που αναγνωρίζει το σύστημα, τροποποιούν και προσδιορίζουν κάθε μια από αυτές, αποτελούν σημασιολογικούς προσδιοριστές.
Οι υπόλοιπες λέξεις που δεν συνδέονται με κάποια αναγνωρισμένη σημασιολογική έννοια επισημαίνονται με μία κενή κλάση, \noslot{}.

Οι δύο αυτοί στόχοι αποτελούν ακόμη μια δύσκολη πρόκληση για την ερευνητική κοινότητα~\cite{tur2011intent,sarikaya2017technology}.

Η αναγνώριση πρόθεσης συνήθως υλοποιείται με μεθόδους ταξινόμησης κειμένου, κυρίως με μεθόδους μάθησης με επίβλεψη.
Βασικό πρόβλημα αποτελεί η ανάγκη μεγάλου αριθμού επισημασμένων δεδομένων των οποίων η έλλειψη είναι ιδιαίτερα φανερή σε καινούργιους κλάδους~\cite{wang2016recent}.
Συναντώνται υλοποιήσεις που χρησιμοποιούν
τον αλγόριθμο Boosting~\cite{schapire2000boostexter},
\newterm{Μηχανές Διανυσμάτων Υποστήριξης}{Support Vector Machines - SVM}~\cite{haffner2003optimizing,bhargava2013easy},
\newterm{Ταξινομητές Μέγιστης Εντροπίας}{Maximum Entropy Classifiers}~\cite{ang2005automatic} και
μοντέλα λογιστικής παλινδρόμησης~\cite{snips,rasa}.
Σε πιο σύγχρονες δημοσιεύσεις, εμφανίζονται μέθοδοι βασισμένες σε νευρωνικά δίκτυα όπως:
\newterm[Νευρωνικά Δίκτυα]{Βαθιά Δίκτυα Πεποιθήσεων}{Deep Belief Networks - DBN}~\cite{sarikaya2014application},
\newterm[Νευρωνικά Δίκτυα]{Συνελικτικά\dd{ Νευρωνικά Δίκτυα}}{Convolutional\dd{ Neural Networks} - CNN}~\cite{kim2014convolutional,zhang2015sensitivity},
συνδυασμός CNN με \newterm[Νευρωνικά Δίκτυα]{\dd{Δίκτυα }Μακράς Βραχυπρόθεσμης Μνήμης}{Long Short-Term Memory - LSTM}~\cite{zhou2015c} και
\newterm[Νευρωνικά Δίκτυα]{Ανατροφοδοτούμενα\dd{ Νευρωνικά Δίκτυα}}{Recurrent\dd{ Neural Networks} - RNN}~\cite{yang2016hierarchical},

Οι βασικές παραδοσιακές προσεγγίσεις για την εξαγωγή οντοτήτων βασίζονται σε
\newtermprint[Hidden Markov Models - HMM]{Κρυφά Μαρκοβιανά Μοντέλα}~\cite{wang2005spoken} και
CRF~\cite{raymond2007generative,wang2011semantic,wang2016recent,rasa,snips}.
Πιο πρόσφατα, έχουν επίσης χρησιμοποιηθεί RNN~\cite{mesnil2013investigation,yao2013recurrent,mesnil2015using,liu2015recurrent,vu2016bi} και LSTM~\cite{yao2014spoken} δίκτυα.

Η ανάθεση σημασιολογικών ρόλων (\SRLR{}) έχει εμφανισθεί και στη βιβλιογραφία των συστημάτων διαλόγου.
Οι \citet{tur2005semi} χρησιμοποίησαν SRL για την \newtermprint[clustering]{Ομαδοποίηση} προτάσεων χρηστών που δεν είχαν κατηγοριοποιηθεί σύμφωνα με την πρόθεσή τους.
Από τα δεδομένα εκπαίδευσης εξάγονται τα πιο συχνά εμφανιζόμενα ζευγάρια δομών κατηγορημάτων-ορισμάτων και από αυτά ένας ειδικός καλείται να δημιουργήσει κανόνες που μεταφράζουν τις δομές αυτές σε προθέσεις.
% TODO: περιορισμός που ισχύει και σε εμάς: δεν γίνεται ανίχνευση προθέσεων που δεν περιέχουν ρήμα.
% Αυτό αποτελεί σημαντικότερο πρόβλημα στα συστήματα διαλόγου όπου είναι πιο συνήθης η παράλειψη ρημάτων με εντολές όπως \engquote{account balance}.
Στο~\cite{chen2013unsupervised} επιχειρείται η αυτόματη εξαγωγή και πλήρωση σημασιολογικών υποδοχέων ακατέργαστων ηχητικών δεδομένων για την επιτάχυνση της διαδικασίας δημιουργίας συστήματος διαλόγου ομιλίας.

\subsubsection{Κοινά μοντέλα}
Στη βιβλιογραφία προτείνονται \newtermprint[Joint Models]{Κοινά Μοντέλα}
που στοχεύουν στην ταυτόχρονη επίλυση των προβλημάτων της αναγνώρισης προθέσεων και εξαγωγής οντοτήτων.
Η λογική αυτών των προσεγγίσεων βασίζεται στο ότι υπάρχει ισχυρή στατιστική διασύνδεση των δύο αυτών διεργασιών.
Αποτελούν προσεγγίσεις που προσπαθούν να συνδυάσουν την επισήμανση μιας ακολουθίας κειμένου με την ταξινόμησή της.
Δηλαδή, το μοντέλο πρέπει να καταχωρεί μια ετικέτα σε κάθε στοιχείο της ακολουθίας και να ταξινομεί την ακολουθία ως σύνολο με μια κλάση ή ετικέτα.
Μια προσέγγιση που αναφέρεται στη βιβλιογραφία είναι τα
\newtermc[Υπό Συνθήκη Τυχαίο Πεδίο - Conditional Random Field - CRF]{\rr{τ}{Τ}ριγωνικ\rr{ά CRF}{ό}}{Triangular\dd{ CRF} - TriCRF}~\cite{jeong2008triangular}
που επεκτείνουν τη λειτουργία των γραμμικών \CRFR{} χρησιμοποιώντας μια επιπλέον μεταβλητή που εκφράζει την πρόθεση της πρότασης εισόδου.
Στο~\cite{xu2013convolutional} προτάθηκε μια έκδοση των TriCRF με CNN όπου τα χαρακτηριστικά που χρησιμοποιούνται εξάγονται από τις βαθμίδες του νευρωνικού δικτύου.
Επίσης, έχουν χρησιμοποιηθεί LSTM δίκτυα~\cite{hakkani2016multi,zhou2016hierarchical}.

Τέλος, μερικές προσεγγίσεις προσπαθούν να αξιοποιήσουν την επιτυχία της \newterm[Νευρωνικά Δίκτυα]{Μεταφορά\dd{ς} Μάθησης}{Transfer Learning} σε άλλα προβλήματα,
όπως στον τομέα της μηχανικής όρασης~\cite{sharif2014cnn,girshick2014rich,donahue2014decaf}.
Η πρόσφατη δημοτικότητα δημοσιεύσεων που σχετίζονται με έννοιες μεταφοράς μάθησης όπως τα BERT~\cite{bert}, ELMO~\cite{elmo} και ULMFiT~\cite{ulmfit} στον κλάδο της επεξεργασίας φυσικής γλώσσας
καταδεικνύει ότι είναι πιθανό να εμφανιστούν αρκετές μελλοντικές εφαρμογές με μοντέλα που εκμεταλλεύονται τη μεταφορά πληροφορίας.
Στο~\cite{goyal2018fast} χρησιμοποιούνται
\newterm[Νευρωνικά Δίκτυα!Μακράς Βραχυπρόθεσμης Μνήμης - Long Short-Term Memory - LSTM]{Αμφίδρομα \dd{δίκτυα μακράς βραχυπρόθεσμης μνήμης}}{Bi-directional\dd{ LSTM Networks}}
που εκπαιδεύονται πρώτα σε σύνολα δεδομένων που περιέχουν μεγάλο αριθμό επισημασμένων προθέσεων και οντοτήτων και μετά προσαρμόζουν το δίκτυο σε ένα καινούργιο σύνολο δεδομένων με μικρότερο αριθμό παραδειγμάτων.

\subsubsection{Multi-intent detection}\label{subsec:multi-intent}
% TODO: μάλλον αλλού για να σχολιάσω τις διαφορές μας με αυτούς;
% TODO: να αιτιολογήσω SRL -> intent σχέση μέσω βιβλιογραφίας. Επίσης, το πρόβλημα είναι ότι αυτό προυποθέτει intents που
%εκφράζονται μόνο μέσω ρημάτων. Πχ δε γίνεται "Awesome" όπως στο https://github.com/RasaHQ/tutorial-tf-pipeline. Αυτό,
%θα μπορούσε είτε να αντιμετοπισθεί με ίδια λογική που χρησιμοποιούν τα \cite{xu2013exploiting,RasaBlog}: να θεωρούμε
%ξεχωριστά labels τους συνδυασμούς που δεν ανιχνεύονται είτε να αγνοηθούν και να αναγνωρίζεται μόνο η \enquote{κυρίαρχη}, όπως
%γίνεται στα περισσότερα συστήματα διαλόγου~\cite{TODO}. Πάντως, στη δική μας εφαρμογή δεν υπάρχουν προθέσεις που να
%πέφτουν σε αυτή την κατηγορία και το πρόβλημα δεν αντιμετωπίζεται.
Στα περισσότερα σύγχρονα συστήματα διαλόγου προσανατολισμένα για καθήκοντα, θεωρείται ότι για κάθε είσοδο ή πρόταση του χρήστη υπάρχει μόνο μια πρόθεση.
Το υποπρόβλημα της
\newterm[Αναγνώριση Πρόθεσης - Intent Identification]{Αναγνώριση\dd{ς} Πολλαπλών Προθέσεων}{Multi-Intent / Multiple Intent Identification}
ανά πρόταση είναι ένας τομέας με συγκριτικά λίγες σχετικές δημοσιεύσεις.
Για παράδειγμα, στην πρόταση \engquote{Move forwards while waving hand} πρέπει να ανιχνευθούν οι προθέσεις \intent{BodyMotion} και \intent{ArmMotion}.
Τέτοιου τύπου προτάσεις μπορούν να δυσχεράνουν μοντέλα που είναι εκπαιδευμένα σε δεδομένα που περιέχουν αποκλειστικά \newtermprint[Single Intent]{μονές προθέσεις} ανά πρόταση.

Οι \citet{xu2013exploiting} προσπαθούν να επιλύσουν το πρόβλημα χρησιμοποιώντας κατηγοριοποίηση πολλών ετικετών.
Ωστόσο, το μοντέλο τους απαιτεί την εκπαίδευση πάνω σε δεδομένα με προτάσεις που περιέχουν πολλαπλές προθέσεις χρηστών.
Αυτό αυξάνει σημαντικά τις απαιτήσεις για δεδομένα εκπαίδευσης.
Επίσης, οι πιθανές προθέσεις ανά πρόταση περιορίζονται σε δύο και οι συνδυασμοί τους θεωρούνται ως ξεχωριστή ετικέτα,
για παράδειγμα \intent{buy_game+play_game}.

Οι \citet{kim2017two} επιχειρούν την ανίχνευση πολλαπλών προθέσεων σε δύο στάδια.
Στο πρώτο, η πρόταση χωρίζεται σε πιθανές υποθέσεις σύμφωνα με τους \newterm{\rr{συνδέσμους}{Σύνδεσμος}}{Conjunction} που περιέχονται στην πρόταση.
Για παράδειγμα, η πρόταση \engquote{Record Phineas and Ferb and play OCN news} παράγει δύο υποθέσεις:
\begin{compactenum}
    \item \engquote{Record Phineas and Ferb} \textbf{and} \engquote{play OCN news}
    \item \engquote{Record Phineas} \textbf{and} \engquote{Ferb and play OCN news}
\end{compactenum}
όπου, προφανώς, ο πρώτος διαχωρισμός είναι ο επιθυμητός.
Σε κάθε υπόθεση δίνεται η ίδια βαθμολογία με την ελάχιστη βαθμολογία των προθέσεων που την αποτελούν.
Επιλέγεται η υπόθεση με τη μέγιστη βαθμολογία.
Το δεύτερο στάδιο κυρίως αφορά προτάσεις στις οποίες
η \newtermprint[Automatic Speech Recognition - ASR]{Αυτόματη Αναγνώριση Ομιλίας}
αποτυγχάνει να αναγνωρίσει τους απαραίτητους συνδέσμους και η αναγνώριση των πολλαπλών προθέσεων γίνεται μέσω μιας προσέγγισης επισήμανσης ακολουθίας.
Ένα πλεονέκτημα αυτής της προσέγγισης είναι η έλλειψη της ανάγκης ύπαρξης δεδομένων εκπαίδευσης που να περιέχουν πολλές προθέσεις ανά πρόταση.
Το πρώτο στάδιο χρησιμοποιεί μοντέλα εκπαιδευμένα μόνο σε μονές προθέσεις ανά πρόταση
ενώ το δεύτερο χρησιμοποιεί μια μέθοδο αυτόματης παραγωγής κατάλληλων δεδομένων με πολλαπλές προθέσεις ανά είσοδο.
Ένας σημαντικός περιορισμός είναι ότι λειτουργεί με την προϋπόθεση ότι ο μέγιστος αριθμός προθέσεων ανά πρόταση είναι δύο.
Επίσης, η μέθοδος που ακολουθείται αδυνατεί να αναλύσει επιτυχώς πιο περίπλοκες συντακτικές δομές όπως η ανύψωση δεξιού κόμβου (βλέπε \fref{subsec:linguistics}).

Η χρήση ενός \newtermprint[Hierarchical Model]{Ιεραρχικού Μοντέλου} προτείνεται από τους \en{\citet{rychalska2018multi}}.
Αρχικά, πραγματοποιείται η κατηγοριοποίηση των \enquote{μεγάλων υποδοχών} (\en{Big Slots}) οι οποίες περιέχουν το τμήμα της πρότασης που περιλαμβάνει την πρόθεση του χρήστη.
Στη συνέχεια, πραγματοποιείται η κατηγοριοποίηση των λεπτομερειών --- \enquote{μικρών υποδοχών} (\en{Small Slots}) --- με βάση τα χαρακτηριστικά της μεγάλης υποδοχής στην οποία εμπεριέχεται η μικρή.
Δοκιμάζεται ένα μοντέλο με CRF και ένα με δίκτυα GRU-RNN.
Για τους σκοπούς της εκπαίδευσης δημιουργήθηκε μια βάση δεδομένων που αποτελείται από προτάσεις που ενίοτε περιέχουν πολλαπλές προθέσεις.
% TODO: πλεονεκτήματα μας - SRL είναι πιο γενικό και δεν απαιτεί εκπαίδευση σε αρκετά δεδομένα.
% Η μέθοδος της παραγωγής μπορεί να μην ανταποκρίνεται με μεγάλη ευρωστία σε πιο περίπλοκες συντακτικές δομές.
% Ωστόσο, η χρήση RNN σε αρκετά μεγάλες βάσεις δεδομένων ίσως μπορέσει να προσφέρει καλύτερα αποτελέσματα καθώς το
% αποτέλεσμα δεν θα εξαρτάται από τη σημασιολογική δομή της πρότασης.

Οι \citet{xia2018zero} προτείνουν, χωρίς να έχουν υλοποιήσει, την εύρεση πολλαπλών προθέσεων μέσω του διαχωρισμού τους σε απλές προθέσεις και την επισήμανση αυτών με εργαλεία ακολουθιακής κατηγοριοποίησης όπως τα CRF.
Ένα πιθανό πρόβλημα με αυτή την προσέγγιση είναι ότι είναι δυνατή η επικάλυψη των προθέσεων μέσα σε μια πρόταση.
Δηλαδή, μια λέξη μιας πρότασης θα μπορούσε να ανήκει ταυτόχρονα σε παραπάνω από μια πρόθεση.
Μια λύση σε αυτό το πρόβλημα θα μπορούσε να είναι η χρήση μοντέλων πολλών ετικετών.

\subsubsection{Υλοποιήσεις σε βιβλιοθήκες λογισμικού ανοιχτού κώδικα}
\ig[type=pdf,inc={width=\linewidth}]{snips}{\lcaption{Μονάδα NLU του \lib{Snips}}{%
        Συνδυάζονται δύο προσεγγίσεις για την αναγνώριση της πρόθεσης~\protect\cite{snips}.
        Ο \newtermprint[Deterministic Intent Parser]{Ντετερμινιστικός Αναλυτής Πρόθεσης}
        δημιουργεί κανόνες κανονικών εκφράσεων έτσι ώστε να είναι σίγουρη η σωστή ταξινόμηση των παραδειγμάτων που συναντώνται στο σύνολο εκπαίδευσης.
        Ο \newtermprint[Probabilistic Intent Parser]{Πιθανοτικός Αναλυτής Πρόθεσης}
        αποτελείται από ένα μοντέλο λογιστικής παλινδρόμησης για την αναγνώριση πρόθεσης και ένα γραμμικό CRF για τις οντότητες κάθε πρόθεσης.
        Δηλαδή, αφού αναγνωρισθεί η πρόθεση, το αντίστοιχο CRF καλείται να εξάγει τις οντότητες του δοθέντος κειμένου.

        Τέλος, οι τιμές των οντοτήτων μετατρέπονται από φυσική γλώσσα σε τιμές του μεγέθους που μετράνε μορφοποιημένες σύμφωνα με ISO.
        Για παράδειγμα, το \engquote{tomorrow} θα μετατραπεί σε μορφή \en{\texttt{YYYY-MM-DD hh:mm:ss}}.
        Για την εξαγωγή των οντοτήτων χρησιμοποιείται η βιβλιοθήκη \libcite{Rustling} που αποτελεί μια εκ νέου υλοποίηση της βιβλιοθήκης \libcite[Python!]{Duckling}.%
    }%
}
Το \libcite[Python!]{Snips} αποτελεί μια πλατφόρμα που στοχεύει στη δημιουργία ψηφιακών βοηθών που αλληλεπιδρούν φωνητικά με τον χρήστη.
Χρησιμοποιεί μια \newtermprint[Software Pipeline]{σωλήνωση λογισμικού}, που παρουσιάζεται στο \fref{fig:snips}, για την υλοποίηση της μονάδας κατανόησης φυσικής γλώσσας%
\footnote{Ο κώδικας της μονάδας βρίσκεται στο \url{https://github.com/snipsco/snips-nlu}}.
Στη δημοσίευση αιτιολογείται η επιλογή των CRF αντί για τα μοντέλα RNN των~\cite{mesnil2013investigation,mesnil2015using} διότι είναι υπολογιστικά πιο ελαφριά με παρόμοιες ωστόσο αποδόσεις.
Παρόλα αυτά, χρήσιμη θα ήταν η διερεύνηση της χρήσης νεότερων μοντέλων από πιο πρόσφατες δημοσιεύσεις.

Στο \libcite[Python!]{RASA} προσφέρεται ένα ζευγάρι εργαλείων
για την κατανόηση φυσικής γλώσσας\footnote{Ο κώδικας βρίσκεται στο \url{https://github.com/RasaHQ/rasa}}
και για τη διαχείριση διαλόγου\footnote{Ο κώδικας βρίσκεται στο \url{https://github.com/RasaHQ/rasa_core}}.
Η πρώτη αποτελεί μια βιβλιοθήκη που προσφέρει διάφορες μονάδες σχετικές με την κατανόηση και επεξεργασία φυσικής γλώσσας.
Στόχος είναι η δημιουργία μιας σωλήνωσης λογισμικού σύμφωνα με τις προτιμήσεις του χρήστη / προγραμματιστή.
Ως προεπιλογή για την αναγνώριση πρόθεσης, χρησιμοποιείται ένα γραμμικό SVM πολλών κλάσεων με τη βοήθεια της βιβλιοθήκης \libcite[Python!]{scikit-learn}.
Ως είσοδος μπορούν να δοθούν \newterm{Διανύσματα Λέξεων}{Word Vectors} όπως παράγονται από το \libcite{GloVe} ή οποιαδήποτε άλλη μέθοδο αναπαράστασης λέξεων με αριθμητικά διανύσματα.
Εναλλακτικά, μπορεί να γίνει χρήση ενός μοντέλου νευρωνικών δικτύων, με τη βοήθεια του \libcite[Python!]{Tensorflow},
που βασίζεται στην ιδέα από το \lib{StarSpace}~\cite{wu2018starspace} και επιτρέπει την αναγνώριση πολλαπλών προθέσεων αν ο προγραμματιστής συμπεριλάβει δεδομένα για κάθε επιθυμητό συνδυασμό προθέσεων.
Για την εξαγωγή προκαθορισμένων οντοτήτων όπως απόσταση και ημερομηνίες υπάρχει και εδώ μονάδα που χρησιμοποιεί τη βιβλιοθήκη \libcite[Python!]{Duckling}.
Για μη-προκαθορισμένες οντότητες που εισάγει ο προγραμματιστής γίνεται χρήση CRF.

Το \libcite[Python!]{DeepPavlov} αποτελεί μια βιβλιοθήκη που χρησιμοποιεί τεχνικές βαθιάς μάθησης για τη δημιουργία συστημάτων διαλόγων προσανατολισμένων και μη για καθήκοντα.
Προσφέρονται διεπαφές για τη χρήση διάφορων αρχιτεκτονικών νευρωνικών δικτύων και είναι η δυνατή η χρήση των ELMO~\cite{elmo} και BERT~\cite{bert}.

% vim:ts=4:sw=4:expandtab:fo-=tc:tw=120

\section{Σημασιολογική ανάλυση}
Μια άλλη κατεύθυνση που συναντάται στη βιβλιογραφία για σκοπούς σχετικούς με αυτούς της παρούσας διπλωματικής σχετίζεται με τη \newterm{Σημασιολογική ανάλυση}{Semantic Parsing} κειμένων.

\subsection{Ρομποτικές εφαρμογές}
Το πρόβλημα της \newterm{Θεμελίωση\dd{ς} Φυσικής Γλώσσας}{Grounding Natural Language} σε έρευνες του τομέα της ρομποτικής συνήθως επιλύεται λαμβάνοντας υπόψη το περιβάλλον του ρομπότ.
Αυτό αποτελεί σημαντική διαφοροποίηση από τους στόχους αυτής της εργασίας όπου γίνεται προσπάθεια γενίκευσης του προβλήματος της παραγωγής κώδικα ρομποτικής εφαρμογής,
διατηρώντας ως κοινή αναφορά μόνο το μέτα-μοντέλο \metamodel{}.

Πολλές δημοσιεύσεις επιχειρούν τη μετατροπή εντολών φυσικού κειμένου στη \newtermprint[Robot Control Language - RCL]{Γλώσσα Ρομποτικού Ελέγχου}.
Στο~\cite{matuszek2013learning} γίνεται εκμάθηση ενός αναλυτή μέσω της εκπαίδευσης σε ζευγάρια εντολών γραμμένες στα αγγλικά και της μετάφρασής τους σε RCL.
Τα κείμενα αφορούν οδηγίες δρομολόγησης ενός ρομπότ μέσα σε έναν προηγουμένως άγνωστο κλειστό χώρο.
Χρησιμοποιείται \newtermprint[Combinatory Categorial Grammar - CCG]{Συνδυαστική κατηγοριακή γραμματική} που μοντελοποιεί ταυτόχρονα τη συντακτική και σημασιολογική δομή μιας πρότασης.
Οι \citet{shimizu2009learning} εκπαιδεύουν ένα Μαρκοβιανό μοντέλο κατανόησης φυσικής γλώσσας για την ανάλυση οδηγιών δρομολόγησης.
Επίσης, έχουν χρησιμοποιηθεί μέθοδοι ενισχυτικής μάθησης~\cite{branavan2009reinforcement}
και πιθανοτικοί γράφοι~\cite{tellex2011understanding,tellex2011approaching}
για την αντιστοίχιση γλώσσας σε εντολές.

\subsection{Άλλες εφαρμογές}
Η σημασιολογική ανάλυση μπορεί να χρησιμοποιηθεί στην παραγωγή κώδικα δεδομένης μιας λεκτικής περιγραφής.
Στο~\cite{quirk2015language} γίνεται εξαγωγή \enquote{συνταγών} που χρησιμοποιούνται στη διαδικτυακή υπηρεσία \lib{IFTTT}\footnote{\url{https://ifttt.com/}}.
Τα~\cite{yin2017syntactic,rabinovich2017abstract} εκμεταλλεύονται νευρωνικά δίκτυα για την παραγωγή κώδικα \lib{Python} από φυσική γλώσσα.

\subsubsection{Παράδειγμα από τεχνολογία λογισμικού}
Οι \citet{diamantopoulos2017software} προσπαθούν να αυτοματοποιήσουν την αντιστοίχιση λειτουργικών απαιτήσεων,
που γράφονται σε φυσική γλώσσα στα πρώιμα στάδια της ανάπτυξης λογισμικού,
σε \newtermprint[Formal Specification]{τυπικές προδιαγραφές}.
Δεδομένης μιας πρότασης που περιέχει μια λειτουργική απαίτηση, γίνεται εξαγωγή των οντοτήτων και των σχέσεων μεταξύ τους,
είτε αυτών που εκφράζονται ρητά είτε εκείνων που μπορούν να συναχθούν.

Το μοντέλο τους περιλαμβάνει μια σωλήνωση λογισμικού που εκτελεί συντακτική ανάλυση, σημασιολογική ανάλυση, εξαγωγή χαρακτηριστικών και εκπαίδευση ταξινομητών λογιστικής παλινδρόμησης.
Κατά τη σημασιολογική ανάλυση, γίνεται κατηγοριοποίηση των όρων του κειμένου στις αντίστοιχες οντολογικές έννοιες ή ιδιότητες.

Αν και βρίσκεται αρκετά κοντά στους στόχους αυτής της διπλωματικής εργασίας, θεωρείται ότι οι προτάσεις των λειτουργικών απαιτήσεων είναι διατυπωμένες σε σχετικά επίσημη και δομημένη γλώσσα.
Επίσης, η οντολογία τους είναι περιορισμένη και περιγράφει μόνο γενικές έννοιες όπως για παράδειγμα \enquote{\entity{Πράκτορας}} (\entity{Actor}), \enquote{\entity{Αντικείμενο}} (\entity{Object}) κ.ά.,
με περιορισμένο αριθμό υποκλάσεων.

% vim:ts=4:sw=4:expandtab:fo-=tc:tw=120


% vim:ts=4:sw=4:expandtab:fo-=tc:tw=120

\chapter{Μεθοδολογία}\label{chap:methodology}

Στο παρόν κεφάλαιο περιγράφεται το σύστημα ανάλυσης ρομποτικών εντολών \projectname{} που αναπτύχθηκε στο πλαίσιο αυτής της διπλωματικής εργασίας.
Ο κώδικας του συστήματος βρίσκεται στο \url{https://github.com/robotics-4-all/2017_B_NLP_robotics}.
Είναι γραμμένος σε \en{Python 3.7}\footnote{\url{https://www.python.org/}} (μπορεί να τρέξει και σε \en{Python 3.6} με μικρές τροποποιήσεις)
και \en{Bash}\footnote{\url{https://www.gnu.org/software/bash/}}.
Για την τεκμηρίωση του συστήματος χρησιμοποιείται \LaTeX{}\footnote{\url{https://www.latex-project.org/}} και \en{Markdown}\footnote{\url{https://daringfireball.net/projects/markdown/}}.

\section{Βιβλιοθήκες}
\newcommand{\libitem}[2][]{%
    \begingroup%
    \let\lib\enbfttt%
    \item\libcite[#1]{#2}:%
    \endgroup%
}%
Διάφορες βιβλιοθήκες ανοιχτού λογισμικού χρησιμοποιήθηκαν για την υλοποίηση της παρούσας διπλωματικής.
Αναφέρονται σε αυτή την ενότητα συγκεντρωτικά και θα γίνεται πιο συγκεκριμένη αναφορά όπου κρίνεται απαραίτητο.
\begin{compactitem}
    \libitem{CoreNLP} Εργαλειοθήκη επεξεργασίας φυσικής γλώσσας του \en{Stanford}, γραμμένη σε \en{Java}.
    Προσφέρει διάφορα εργαλεία για την επεξεργασία και ανάλυση φυσικής γλώσσας και εμπεριέχει διάφορα μοντέλα που επιλύουν προβλήματα πέραν των βασικών.
    \libitem[Python!]{spaCy} Βιβλιοθήκη επεξεργασίας φυσικής γλώσσας σε \en{Python}, σχεδιασμένη με στόχο κυρίως την επαγγελματική χρήση.
    Παρέχει υπολογιστικά αποδοτικές λύσεις για διάφορες βασικές λειτουργίες στην επεξεργασία φυσικής γλώσσας.
    \libitem[Python!]{Snips} Πλατφόρμα δημιουργίας ψηφιακών βοηθών σε \en{Python}.
    Βλέπε και σχήμα \ref{fig:snips} στη σελίδα~\pageref{fig:snips}.
    \libitem[Python!]{neuralcoref} Επέκταση για τη \en{spaCy} που πραγματοποιεί επίλυση συναναφορών.
    \libitem[Python!]{AllenNLP} Ερευνητική βιβλιοθήκη επεξεργασίας φυσικής γλώσσας σε \en{Python}.
    Προσφέρει μοντέλα που επιλύουν δύσκολα προβλήματα επεξεργασίας και κατανόησης φυσικής γλώσσας.
    Συνήθως βασίζονται σε αρχιτεκτονικές νευρωνικών δικτύων.
    \libitem[Python!]{DeepPavlov} Πλατφόρμα δημιουργίας ψηφιακών βοηθών που βασίζεται στη χρήση βαθιών νευρωνικών δικτύων σε \en{Python}.
    \libitem[Python!]{PyEcore} Υλοποίηση της πλατφόρμας μοντελοποίησης του Eclipse (\en{Eclipse Modeling Framework - EMF}) σε \en{Python}.
    \libitem[Python!]{networkx} Βιβλιοθήκη δημιουργίας γράφων και δικτύων σε \en{Python}.
    \libitem[Python!]{matplotlib} Βιβλιοθήκη σχεδίασης δισδιάστατων γραφημάτων με λειτουργία παρόμοια της αντίστοιχης βιβλιοθήκης του \en{MATLAB} σε \en{Python}.
    \libitem{Graphviz} Λογισμικό για την απεικόνιση γράφων.
    \libitem[Python!]{requests} Βιβλιοθήκη της \en{Python} που πραγματοποιεί \newtermprint[Requests]{Αιτήσεις} χρησιμοποιώντας το πρωτόκολλο HTTP.
    \libitem[Python!]{braceexpand} Βιβλιοθήκη της \en{Python} που επεκτείνει χαρακτήρες που περιέχονται μέσα σε μια ομάδα αγκίστρων.
\end{compactitem}

\section{Γραφική γλώσσα \metamodel{}}
Κατά την αναγνώριση των ενεργειών και οντοτήτων που περιλαμβάνονται μέσα στο κείμενο ενός χρήστη απαιτείται η ύπαρξη ενός προκαθορισμένου συνόλου που τις περιλαμβάνει
(βλέπε και ενότητες \ref{sec:problem-description} και \ref{sec:diploma-purpose}).
Για τον σκοπό αυτό, χρησιμοποιήθηκε η γραφική γλώσσα \metamodel{}\footnote{\url{https://r4a.issel.ee.auth.gr/nao4a/}} και το σχετικό μέτα-μοντέλο.

Στόχος της γλώσσας αποτελεί η ανάπτυξη εφαρμογών για το ρομπότ NAO\footnote{\url{https://www.softbankrobotics.com/emea/en/nao}}.
Πρόκειται για ένα ανθρωπόμορφο ρομπότ ύψους 58 εκατοστών σχεδιασμένο από τη \en{Softbank Robotics}.
Διαθέτει πολλούς αισθητήρες και κινητήρες που του δίνουν ελευθερία κινήσεων και τη δυνατότητα να αλληλεπιδρά με το περιβάλλον του.
Ένα σχηματικό διάγραμμα δίνεται στο \fref{fig:nao}.
\ig[type=pdf,inc={width=0.7\linewidth}]{nao}{\caption{Διάγραμμα του ρομπότ NAO}}

Το εργαλείο αυτό αναπτύχθηκε από την ομάδα \en{Robotics 4 All (R4A)}\footnote{\url{https://r4a.issel.ee.auth.gr}}
του Τμήματος Ηλεκτρολόγων Μηχανικών και Μηχανικών Υπολογιστών του Αριστοτελείου Πανεπιστημίου Θεσσαλονίκης.
Έχει στόχο τη διευκόλυνση της ανάπτυξης ρομποτικών εφαρμογών με τη χρήση ενός απλού γραφικού περιβάλλοντος μέσα από το
\newtermprint[Integrated Development Environment - IDE]{Ολοκληρωμένο Περιβάλλον Ανάπτυξης} \en{Eclipse}\footnote{\url{https://www.eclipse.org/}}.
Το περιβάλλον αυτό, μαζί με ένα παράδειγμα ενδεικτικής ρομποτικής εφαρμογής, παρουσιάζεται στο \fref{fig:4-loop}.

Βασίζεται σε μοντελοστραφείς τεχνικές ανάπτυξης λογισμικού και παράγει τον κατάλληλο κώδικα \en{Python} για το λειτουργικό σύστημα \en{NAOqi} με βάση τον γράφο που δημιουργεί ο χρήστης.
Κάθε κόμβος αναπαριστά μία συμπεριφορά του ρομπότ.
Αυτές συνδέονται μεταξύ τους με τις σωστές μεταβάσεις.
Μια συνοπτική παρουσίαση του μέτα-μοντέλου φαίνεται στο \fref{fig:metamodel}.

\ig[pos=t,type=png,inc={width=\linewidth}]{4-loop}{\lcaption{Παράδειγμα χρήσης γραφικής γλώσσας \metamodel{} με βρόχο επανάληψης}{%
        Κατά την εκτέλεση αυτής της εφαρμογής, το NAO θα προχωρήσει ευθεία, διατηρώντας την αρχική του κατεύθυνση, μέχρι να ανιχνεύσει την ύπαρξη ανθρώπου.
        Αφού ανιχνεύσει κάποιον άνθρωπο, θα εκφωνήσει \engquote{I found you!} και θα τερματίσει τη λειτουργία του.
        Η διάρκεια ανίχνευσης διαρκεί πέντε δευτερόλεπτα.
        Κάθε φορά που το NAO αποτυγχάνει να ανιχνεύσει κάποιον, εκφωνεί \engquote{Where are you?}
    }%
}

\ig[pos=t,type=png,inc={width=\linewidth}]{metamodel}{\caption{Το μέτα-μοντέλο \metamodel{}}}

\subsection{Ενέργειες του ρομπότ NAO}
\newcommand{\defintent}[2][]{\def\tmp{#2}%
    \ifthenelse{\isempty{#1}}{%
        \lowercase{\def\tmpurlhash{#2}}%
    }{%
        \lowercase{\def\tmpurlhash{#1}}%
    }%
    \def\tmpurl{https://r4a.issel.ee.auth.gr/nao4a/\#\tmpurlhash}%
    \StrDel{\tmp}{ }[\tmp]%
    \StrDel{\tmpurl}{ }[\tmpurl]%
    \hypersetup{urlcolor=black}% Assuming inside group since we call \item
    \item\href{\tmpurl}{\textbf{#2}}\label{intent:\tmp}:}
Οι ενέργειες που υποστηρίζει η \metamodel{} παρατίθενται στη συνέχεια.
Κάθε ενέργεια επίσης παραμετροποιείται με διάφορες ιδιότητες που δεν αναφέρονται εδώ για συντομία.

\subsubsection{Ενέργειες Κίνησης}
\begin{compactitem}
    \defintent{Head Motion} Ο χρήστης μπορεί να κινήσει το κεφάλι του NAO σε μία από τις πέντε θέσεις: αριστερά, δεξιά, πάνω, κάτω, επιστροφή στην ουδέτερη θέση.
    \defintent{Body Motion} Ο χρήστης μπορεί να κινήσει το NAO ως σύνολο δίνοντάς του παραμέτρους για την κατεύθυνση, την απόσταση ή τη διάρκεια της κίνησης, την ταχύτητα και την ελάχιστη απόσταση από εμπόδια.
    \defintent{Arm Motion} Ο χρήστης μπορεί να κινήσει το ένα ή και τα δύο χέρια του ρομπότ δηλώνοντας τον τύπο της κίνησης.
    \defintent{Body Stance} Ο χρήστης μπορεί να επιλέξει τη στάση που θα πάρει το ρομπότ.
    \defintent{Learn Motion} Ο χρήστης μπορεί να μάθει στο NAO μία νέα κίνηση, κινώντας τις αρθρώσεις όπως επιθυμεί και δίνοντάς της ένα όνομα.
    \defintent{Replay Motion} Το ρομπότ μπορεί να αναπαραγάγει μία κίνηση που έμαθε προηγουμένως από τον χρήστη χρησιμοποιώντας το όνομα που της δόθηκε.
\end{compactitem}

\subsubsection{Αλληλεπίδραση με τον χρήστη}
\begin{compactitem}
    \defintent{Detect Touch} Ο χρήστης μπορεί να ρυθμίσει το ρομπότ ώστε αυτό να είναι έτοιμο να αναγνωρίσει άγγιγμα σε κάποιο μέρος ή μέρη του σώματός του.
    \defintent{Detect Human} Ο χρήστης μπορεί να ρυθμίσει το ρομπότ ώστε αυτό να είναι έτοιμο να αναγνωρίσει την ύπαρξη ανθρώπινου προσώπου.
    \defintent{Detect Sound} Ο χρήστης μπορεί να ρυθμίσει το ρομπότ ώστε αυτό να είναι έτοιμο να αντιληφθεί την ύπαρξη ήχου.
    \defintent{Detect Motion} Ο χρήστης μπορεί να ρυθμίσει το ρομπότ ώστε αυτό να είναι έτοιμο να αναγνωρίσει την ύπαρξη κίνησης.
    \defintent{Listen} Ο χρήστης μπορεί να ρυθμίσει το ρομπότ ώστε αυτό να είναι έτοιμο να ακούσει ομιλία και να αναγνωρίσει τι είπε ο χρήστης από μία λίστα εκφράσεων.
    \defintent{Record Sound} Ο χρήστης μπορεί να ρυθμίσει το ρομπότ ώστε αυτό να καταγράψει ήχο από το περιβάλλον του για μία συγκεκριμένη διάρκεια.
    \defintent{Replay Sound} Ο χρήστης μπορεί να ρυθμίσει το ρομπότ ώστε αυτό να αναπαράγει κάποιον ήχο που κατέγραψε σε προηγούμενη φάση.
    \defintent{Weather Report} Ο χρήστης μπορεί να ρυθμίσει το ρομπότ ώστε αυτό να είναι έτοιμο να αναφέρει τις καιρικές συνθήκες για μία πόλη από το σύνολο που ορίζει ο χρήστης χρησιμοποιώντας πληροφορίες από το διαδίκτυο.
\end{compactitem}

\subsubsection{Λοιπές Ενέργειες}
\begin{compactitem}
    \defintent{Talk} Ο χρήστης μπορεί να καταχωρήσει μία φράση την οποία θα εκφωνήσει το NAO.
    \defintent{Dice} Δίνει τη δυνατότητα μετάβασης σε μία ενέργεια από ένα σύνολο ενεργειών με τυχαίο τρόπο.
    \defintent{Sleep} Θέτει την εφαρμογή ή μέρος αυτής σε κατάσταση αναμονής για κάποιο χρονικό διάστημα.
    \defintent[ledon]{Turn Led On} Ο χρήστης μπορεί να επιλέξει ποια LEDs θα ενεργοποιηθούν και ποιο χρώμα θα λάβουν.
    \defintent[ledoff]{Turn Led Off} Ο χρήστης μπορεί να επιλέξει ποια LEDs θα απενεργοποιηθούν.
    \defintent{Counter} Ο χρήστης μπορεί να δημιουργήσει ακέραιους μετρητές ή συγκριτές.
\end{compactitem}

\section{Δεδομένα εκπαίδευσης}
Για τα διάφορα μοντέλα μηχανικής μάθησης που χρησιμοποιήθηκαν στο \projectname{} δεν απαιτούνται δεδομένα εκπαίδευσης καθώς επιλύουν γενικά προβλήματα και είναι ήδη εκπαιδευμένα.
Εξαίρεση αποτελούν αυτά που βρίσκονται στη μονάδα \NLU{} που χρειάζεται να προσαρμοστούν πάνω στο συγκεκριμένο στόχο της διπλωματικής.
Για την εκπαίδευση τους δημιουργήθηκε ένα νέο σύνολο δεδομένων με προτάσεις που αντιστοιχούν σε κάθε ενέργεια του \metamodel{}.

Το σύνολο δεδομένων αποτελείται από δύο αρχεία ανά πρόθεση.
Το όνομα του πρώτου είναι της μορφής \mintinline{text}{intent_<IntentName>.yaml}\footnote{Χρησιμοποιείται η γλώσσα \lib{YAML} --- \url{https://yaml.org/}}
και περιλαμβάνει τις πληροφορίες που συνδέονται με τη κάθε πρόθεση, όπως το όνομά, τους υποδοχείς που δέχεται και τον τύπο της οντότητάς τους.
Ο τρόπος συγγραφής αυτών των πληροφοριών καθορίζεται από τη βιβλιοθήκη κατανόησης φυσικής γλώσσας \libcite[Python!]{Snips}\footnote{Περιγράφεται στο \url{https://snips-nlu.readthedocs.io/en/latest/dataset.html}},
ένα παράδειγμα παρουσιάζεται στη \fref{lst:intent-yaml}.

Το δεύτερο αρχείο παίρνει όνομα της μορφής \mintinline{text}{utterances_<IntentName>} και περιλαμβάνει παραδείγματα προτάσεων.
Σε κάθε πρόταση χρησιμοποιείται μόνο μια πιθανή ενέργεια (πρόθεση) καθώς η \lib{Snips} δεν υποστηρίζει πολλαπλές προθέσεις ανά πρόταση και η μεθοδολογία που ακολουθείται από το \projectname{} δεν απαιτεί την ύπαρξη τέτοιων προτάσεων στο σύνολο δεδομένων.

Επειδή η δημιουργία ενός συνόλου που καλύπτει όλα τα σενάρια χρήσης με αρκετή ποικιλομορφία στα κείμενα εισόδου αποτελεί απαιτητική διαδικασία,
τα δεδομένα εκπαίδευσης δεν είναι το κύριο επίκεντρο αυτής της εργασίας και για αυτό δεν είναι διεξοδικά.
Ωστόσο, για την ενίσχυσή του χρησιμοποιήθηκε η βιβλιοθήκη \libcite[Python!]{braceexpand} που επεκτείνει χαρακτήρες που βρίσκονται μέσα σε μια ομάδα αγκίστρων\footnote{\url{https://rosettacode.org/wiki/Brace_expansion} για παραδείγματα με υλοποιήσεις σε διάφορες γλώσσες προγραμματισμού}.
Για παράδειγμα, η ακολουθία \mintinline[]{text}|It{{em,alic}iz,erat}e{d,}| μετατρέπεται σε μία λίστα με τα εξής στοιχεία:
\mintinline[]{text}|Itemized|,
\mintinline[]{text}|Itemize|,
\mintinline[]{text}|Italicized|,
\mintinline[]{text}|Italicize|,
\mintinline[]{text}|Iterated|,
\mintinline[]{text}|Iterate|.
Κάθε ένας από τους συνδυασμούς που επεκτείνονται χρησιμοποιείται ως ξεχωριστό δεδομένο εκπαίδευσης.
Ένα παράδειγμα της μορφής αυτών των προτάσεων παρουσιάζεται στη \fref{lst:utterances}.

\begin{listing}
    \inputminted{yaml}{../data/intent_ArmMotionOpen.yaml}
    \caption[Δήλωση της πρόθεσης \texttt{ArmMotionOpen}]{Δήλωση της πρόθεσης \intent{ArmMotionOpen}.
        Σημειώνεται ότι η \intent{ArmMotionOpen} δεν αποτελεί ξεχωριστή ενέργεια του \metamodel{} αλλά είναι η \intent{ArmMotion} με \entity{armMotion=OPEN}.
        Αυτή η διάσπαση αναφέρεται στην \fref{subsec:4-snips}.%
    }\label{lst:intent-yaml}
    \inputminted{text}{../data/utterances_ArmMotionOpen}
    \caption[Δεδομένα εκπαίδευσης για τη πρόθεση \texttt{ArmMotionOpen}]{Δεδομένα εκπαίδευσης για τη πρόθεση \intent{ArmMotionOpen}.
        Χρησιμοποιούνται διάφορα ρήματα που αντιστοιχούν σε αυτή την ενέργεια και, όπου είναι δυνατό, ομαδοποιούνται σε αγκύλες.%
    }\label{lst:utterances}
\end{listing}

\section{Μονάδες} % TODO: mention optional
Το~\fref{fig:system} παρουσιάζει συνοπτικά τη σωλήνωση λογισμικού που αναπτύχθηκε.
Στις επόμενες ενότητες θα αναλυθεί η λειτουργία της κάθε μονάδας και η μεθοδολογία που χρησιμοποιήθηκε για την υλοποίησή της και την ενσωμάτωσή της με τις υπόλοιπες.
\ig[type=tikz]{system}{\caption{Σύνοψη συστήματος \projectname{}}}

\subsection{Επισήμανση παραθέσεων}\label{subsec:4-QuoteAnnotator}
\newcommand{\QUOTESTRING}{\engquote{\texttt{QUOTE}}}
Η πρώτη διεργασία που πραγματοποιείται κατά την είσοδο του κειμένου είναι η εύρεση και αντικατάσταση παραθέσεων.
Δηλαδή, το κείμενο που περιλαμβάνεται μέσα σε εισαγωγικά~(\texttt{"}) αντικαθίσταται με το αλφαριθμητικό \QUOTESTRING{}.

Γίνεται η υπόθεση ότι όλες οι παραθέσεις του κειμένου σε ευθύ λόγο ορθά εμπεριέχονται μέσα σε χαρακτήρες εισαγωγικών.

Για την ανίχνευση του διαστήματος των παραθέσεων μέσα στο κείμενο, χρησιμοποιείται ο \lib{QuoteAnnotator}\footnote{\url{https://stanfordnlp.github.io/CoreNLP/quote.html}} (υπομνηματιστής παραθέσεων) του \libcite{CoreNLP}.
Η λειτουργία της συγκεκριμένης μονάδας είναι ντετερμινιστική αντί να βασίζεται σε στατιστικές μεθόδους επεξεργασίας φυσικής γλώσσας.
Παραθέσεις που είναι διατυπωμένες σε πλάγιο λόγο δεν ανιχνεύονται από τον \lib{QuoteAnnotator}.

Η διαδικασία αποσκοπεί στη διευκόλυνση των επόμενων μοντέλων που πραγματοποιούν κάποια μορφή σημασιολογικής ανάλυσης
και για να αποφεύγεται η επεξεργασία του κειμένου που περιλαμβάνεται μέσα στα εισαγωγικά καθώς αυτό μεταφέρεται αυτολεξεί στην έξοδο του μοντέλου.

Τα αλφαριθμητικά \QUOTESTRING{} αντικαθιστώνται με το αρχικό κείμενο στην έξοδο του μοντέλου και στις τιμές των οντοτήτων στη μονάδα NLU (βλέπε \fref{subsec:4-NLU}).

\begin{code}
    \begin{minted}{json}
{
    "quotes": [
        {
            "id": 0,
            "text": "\"hello Alice\"",
            "beginIndex": 9,
            "endIndex": 22,
            "beginToken": 2,
            "endToken": 5,
            "beginSentence": 0,
            "endSentence": 0,
            "speaker": "Unknown",
            "canonicalSpeaker": "Unknown"
        }
    ]
}
    \end{minted}
    \lcaption{Παράδειγμα εξόδου \lib{QuoteAnnotator} σε \en{JSON}}{%
        Είσοδος: \engquote{\texttt{Bob said "hello Alice" and left}}.
        Η δυνατότητα ανίχνευσης ομιλητή απενεργοποιήθηκε για αυτό το παράδειγμα.%
    }
\end{code}

\subsection{\lib{SpaCy}}\label{subsec:4-spacy}
Η βιβλιοθήκη επεξεργασίας φυσικής γλώσσας \libcite[Python!]{spaCy} αποτελεί κεντρικό παράγοντα του \projectname{} αλλά και αρκετών άλλων βιβλιοθηκών από τις οποίες εξαρτάται.
Προσφέρει στατιστικά μοντέλα για διάφορες γλώσσες\footnote{\url{https://spacy.io/usage/models\#languages}} που χρησιμοποιούνται για την επισημείωση γλωσσολογικών χαρακτηριστικών σε σώματα κειμένου.
Παρέχει μια ποικιλία γλωσσικών επισημειώσεων που δίνουν εικόνες σχετικές με τη γραμματική δομή ενός κειμένου.
Αυτή περιλαμβάνει τους τύπους λέξεων, όπως τα μέρη του λόγου, και πώς οι λέξεις σχετίζονται μεταξύ τους.

Η χρήση της βιβλιοθήκης επικεντρώνεται γύρω από τρία βασικά αντικείμενα:
\begin{compactenum}
    \item Το αντικείμενο \python{Token} που είναι μια βασική λεκτική μονάδα --- μπορεί να περιέχει κάποια λέξη, ένα σημείο στίξης, κενό διάστημα κτλ.
    \item Το αντικείμενο \python{Slice} που είναι ένα τμήμα ενός \python{Doc} και περιλαμβάνει ένα συνεχές υποσύνολο από τα \python{Token} που ανήκουν σε αυτό.
    \item Το αντικείμενο εγγράφου \python{Doc} που αποτελείται από μια σειρά από \python{Token}.
          Μπορεί να χρησιμοποιηθεί, μεταξύ άλλων, για την πρόσβαση στις προτάσεις και τις οντότητες του κειμένου.
          Παρόλο που ένα αντικείμενο εγγράφου (\python{Doc}) υποβάλλεται σε επεξεργασία ---
          π.χ.\ χωρίζεται σε μεμονωμένες λέξεις και επισημειώνεται ---
          εξακολουθεί να διατηρεί όλες τις πληροφορίες του αρχικού κειμένου,
          όπως για παράδειγμα τους χαρακτήρες λευκού διαστήματος, επιτρέποντας την ανάκτησή του.
\end{compactenum}
Τα \python{Span} και \python{Token} αποτελούν \newtermprint[Views]{όψεις} των δεδομένων που αποθηκεύονται στο \python{Doc}.
Ένα παράδειγμα χρήσης παρουσιάζεται στην \fref{lst:spacy}.

\begin{listing}
    \begin{minted}{python}
>>> import spacy
>>> nlp = spacy.load("en_core_web_sm")  # Φόρτωση μικρού μοντέλου αγγλικών
>>> doc = nlp("Hello, world. Here are two sentences.")  # αντικείμενο Doc
>>> # Σειριακή πρόσβαση στις λεκτικές μονάδες (for t in doc)
>>> # και ανάγνωση ετικετών μερών του λόγου (t.pos_, t.tag_)
>>> # και συντακτικών εξαρτήσεων (t.dep_)
>>> [(t.text, t.pos_, t.tag_, t.dep_) for t in doc]
[('Hello', 'INTJ', 'UH', 'ROOT'),
 (',', 'PUNCT', ',', 'punct'),
 ('world', 'NOUN', 'NN', 'npadvmod'),
 ('.', 'PUNCT', '.', 'punct'),
 ('Here', 'ADV', 'RB', 'advmod'),
 ('are', 'VERB', 'VBP', 'ROOT'),
 ('two', 'NUM', 'CD', 'nummod'),
 ('sentences', 'NOUN', 'NNS', 'nsubj'),
 ('.', 'PUNCT', '.', 'punct')]
>>> # Αποτέλεσμα αποσαφήνισης ορίων πρότασης
>>> [t.text for t in doc.sents]
['Hello, world.', 'Here are two sentences.']
\end{minted}
    \caption{Παράδειγμα χρήσης \lib{spaCy}}\label{lst:spacy}
\end{listing}

\ig[type=pdf,inc={width=\linewidth}]{spacy-pipeline}{\lcaption{Η σωλήνωση λογισμικού που χρησιμοποιεί το \lib{spaCy} ως προεπιλογή}{
        Το \lib{spaCy} πρώτα κατατέμνει (\en{Tokenizer}) το κείμενο για να δημιουργήσει ένα αντικείμενο εγγράφου \python{Doc}.
        Το \python{Doc} στη συνέχεια υποβάλλεται σε επεξεργασία σε πολλά διαφορετικά βήματα μέσω μιας σωλήνωσης λογισμικού --- αναφέρεται ως \newtermprint[Processing Pipeline]{σωλήνωση επεξεργασίας}.
        Η προεπιλεγμένη σωλήνωση που χρησιμοποιείται αποτελείται από
        έναν \newtermprint[Tagger]{επισημαντή} μερών του λόγου,
        έναν \newtermprint[Parser]{αναλυτή} που αναθέτει τις σχέσεις εξάρτησης μεταξύ των λεκτικών μονάδων
        και ένα μοντέλο \newtermprint[Entity Recognizer]{αναγνώρισης οντοτήτων}.
        Κάθε στοιχείο σωλήνωσης επιστρέφει το επεξεργασμένο \lib{Doc}, το οποίο στη συνέχεια μεταβιβάζεται στο επόμενο στοιχείο.

        Στη θέση των αποσιωπητικών (\ldots{}) μπορούν να χρησιμοποιηθούν επεκτάσεις που προγραμματίζονται από τον χρήστη ή τρίτους.

        Από \url{https://spacy.io/usage/processing-pipelines}.%
    }%
}
\ig[type=pdf,inc={width=\linewidth}]{spacy-architecture}{\lcaption{Σύνοψη της αρχιτεκτονικής του \lib{spaCy}}{Από \url{https://spacy.io/api}.}}

\subsubsection{Κατάτμηση}
Κατά τη διάρκεια της επεξεργασίας, το \lib{spaCy} πρώτα κατατέμνει (\en{Tokenization}) το κείμενο σε λεκτικές μονάδες (\en{Token}),
δηλαδή το χωρίζει σε λέξεις, σημεία στίξης και ούτω καθεξής.
Αυτό γίνεται με την εφαρμογή ειδικών κανόνων για κάθε γλώσσα.
Για παράδειγμα, η στίξη στο τέλος μιας πρότασης θα πρέπει να διαχωριστεί ενώ το ακρωνύμιο \enquote{Η.Π.Α.} θα πρέπει να παραμείνει ενιαία μονάδα.
Κάθε \lib{Doc} αποτελείται από μια συλλογή λεκτικών μονάδων και ο προγραμματιστής μπορεί να τις προσπελάσει σειριακά ή άμεσα με τον δείκτη της μονάδας.

\subsubsection{Επισήμανση}
Μετά την κατάτμηση του κειμένου, το \lib{spaCy} μπορεί να αναλύσει ένα συγκεκριμένο \lib{Doc} και να το επισημάνει με ετικέτες.
Σ' αυτό το σημείο είναι απαραίτητο το στατιστικό μοντέλο, το οποίο επιτρέπει στο \lib{spaCy} να κάνει μια πρόβλεψη για το ποια ετικέτα χρειάζεται σύμφωνα με τα συμφραζόμενα,
για παράδειγμα, η λέξη που ακολουθεί το \engquote{the} στα αγγλικά είναι κατά πάσα πιθανότητα ουσιαστικό.

\subsubsection{Διαχωρισμός προτάσεων στο κείμενο}\label{subsec:4-sents}
Η \newterm{Αποσαφήνιση Ορίων Πρότασης}{Sentence Boundary Disambiguation - SBD} αναφέρεται στο πρόβλημα της εύρεσης και τμηματοποίησης μεμονωμένων προτάσεων.
Στο \lib{spaCy}, τα όρια των προτάσεων υπολογίζονται από το \newtermprint[Syntactic Parse Tree]{συντακτικό δέντρο ανάλυσης},
έτσι ώστε χαρακτηριστικά όπως η στίξη και η κεφαλαιοποίηση να παίζουν σημαντικό αλλά όχι θεμελιώδη ρόλο στον καθορισμό των ορίων της πρότασης.
Συνήθως, αυτό σημαίνει ότι τα όρια της κύριας πρότασης θα συμπίπτουν τουλάχιστον με τα όρια της δευτερεύουσας πρότασης, ακόμη και σε κείμενο με κακή στίξη.

Στο \projectname{} αξιοποιείται αυτός ο διαχωρισμός και πολλές διεργασίες γίνονται μόνο στο πλαίσιο της πρότασης.
Συγκεκριμένα, η ανάθεση σημασιολογικών ρόλων, η εξαγωγή προθέσεων και οντοτήτων και συνεπώς άλλες επόμενες μονάδες υποθέτουν την επεξεργασία κειμένου που περιέχει μόνο μια πρόταση.
Αυτό αποσκοπεί στη βελτίωση της απόδοσης και τη μείωση της πολυπλοκότητας.

Ωστόσο, άλλες διεργασίες πραγματοποιούνται στο σύνολο του κειμένου εισόδου.
Η εύρεση συναναφορών πρέπει να γίνεται σε όλο το κείμενο έτσι ώστε να βρίσκονται συναναφορές στο σύνολό του.
Η επισήμανση και αντικατάσταση παραθέσεων γίνεται επίσης στο σύνολο του κειμένου για να μην επηρεάζεται το μοντέλο συναναφορών
και για να μην γίνεται η ανάλυση του κειμένου που περιλαμβάνεται σε αυτές από το \lib{spaCy}.

\subsection{Επισήμανση συναναφορών}\label{subsec:4-coref-resolution}
Κύριος σκοπός της επισήμανσης συναναφορών στο \projectname{} είναι η ανίχνευση αντωνυμιών και η αντικατάστασή τους με την κύρια αναφορά.
Αυτό βοηθάει στην καλύτερη ανίχνευση των προθέσεων και των οντοτήτων στη μονάδα NLU (βλέπε \fref{subsec:4-NLU}).
Επειδή τα δεδομένα δεν περιλαμβάνουν παραδείγματα που περιέχουν αντωνυμίες,
η επίλυση των αναφορών βελτιώνει την απόδοση των μοντέλων καθώς η νέα είσοδος σε αυτά είναι πιο κοντά στα δεδομένα πάνω στα οποία εκπαιδεύτηκαν.

Διερευνήθηκε η χρήση των βιβλιοθηκών \libcite[Python!]{neuralcoref}, \libcite{CoreNLP} και \libcite[Python!]{AllenNLP}.

\subsubsection{Neuralcoref}\label{subsec:4-neuralcoref}
Αν και το μοντέλο~\cite{lee2017end} του \lib{AllenNLP} είναι πιο σύγχρονο και αναφέρει καλύτερα αποτελέσματα,
προτιμήθηκε να χρησιμοποιηθεί το \lib{neuralcoref} ως βάση λόγω εμπειρικών αποτελεσμάτων στο σύνολο δεδομένων.

Το \lib{neuralcoref} υλοποιήθηκε ως επέκταση της \lib{spaCy}.
Προσθέτει νέες ιδιότητες στα αντικείμενα της \lib{spaCy} που αντιπροσωπεύουν λεκτικά δείγματα (\python{Token}), διαστήματα (\python{Span}) και έγγραφα (\python{Doc}).
Η λειτουργία του βασίζεται στη δημιουργία \newtermprint[Cluster]{συμπλεγμάτων}.
Κάθε σύμπλεγμα αντιπροσωπεύει μια οντότητα που εμφανίζεται στο κείμενο.
Το στοιχείο \python{Cluster.main} είναι το διάστημα της πιο αντιπροσωπευτικής αναφοράς της οντότητας
ενώ η λίστα \python{Cluster.mentions} περιέχει τις υπόλοιπες αναφορές που βρίσκονται στο κείμενο.
Η λίστα των συμπλεγμάτων του κειμένου είναι προσβάσιμη από τα πεδία των αντικειμένων της \lib{spaCy}, για παράδειγμα \python{Doc._.coref_clusters}.

\begin{listing}
    \begin{minted}{python}
>>> import spacy
>>> import neuralcoref
>>> nlp = spacy.load("en_core_web_sm")
>>> neuralcoref.add_to_pipe(nlp)
>>> doc = nlp("My sister has a dog. She loves him")
>>> [t.text for t in doc.sents]
['My sister has a dog.', 'She loves him']
>>> doc._.coref_clusters
[My sister: [My sister, She], a dog: [a dog, him]]
>>> doc._.coref_clusters[1].mentions
[a dog, him]
>>> doc[-1]
him
>>> doc[-1]._.coref_clusters
[a dog: [a dog, him]]
\end{minted}
    \caption{Παράδειγμα χρήσης \lib{neuralcoref}}
\end{listing}

\subsubsection{CorefAnnotator}\label{subsec:4-CorefAnnotator}
Καθώς η επισήμανση συναναφορών αποτελεί δύσκολη διεργασία που μπορεί να αποτύχει σε πολλές περιπτώσεις,
γίνεται και η χρήση του \lib{CorefAnnotator}\footnote{\url{https://stanfordnlp.github.io/CoreNLP/coref.html}} (υπομνηματιστής συναναφορών) του \lib{CoreNLP}.
Προσφέρονται τρεις προσεγγίσεις: μια ντετερμινιστική~\cite{CorefAnnotatorD1,CorefAnnotatorD2,CorefAnnotatorD3}, μια στατιστική~\cite{CorefAnnotatorS} και μια με νευρωνικά δίκτυα~\cite{CorefAnnotatorN1,CorefAnnotatorN2}.
Γίνεται μόνο η χρήση του στατιστικού μοντέλου, το οποίο είναι και το γρηγορότερο σύμφωνα με τις μετρήσεις του \en{Stanford},
αφού η χρήση των τριών διαφορετικών σωληνώσεων που θα απαιτούνταν για τον συνδυασμό όλων των προσεγγίσεων θα ήταν υπολογιστικά χρονοβόρα.
% TODO: code

\subsubsection{Συνδυασμός}\label{subsec:4-coref-combine}
Για την πρόσβαση στις συναναφορές με μια κοινή διεπαφή από τα υπόλοιπα μέρη του συστήματος, γίνεται συγχώνευση των συμπλεγμάτων αναφορών που εξάγονται από τις δύο βιβλιοθήκες.
Χρησιμοποιείται μια προσέγγιση παρόμοια με αυτή που ακολουθεί το \lib{neuralcoref}, δηλαδή οι συναναφορές ενσωματώνονται στα αντικείμενα που χρησιμοποιεί το \lib{spaCy}.

Αντί να χρησιμοποιούνται μέθοδοι εσωτερικές στον κώδικα του \lib{neuralcoref} για τη δημιουργία συμπλεγμάτων,
η τρέχουσα μονάδα απομιμείται τη λειτουργικότητα των διεπαφών του \lib{neuralcoref} που χρησιμοποιούνται και σε άλλα σημεία του κώδικα του \projectname{}.
Κάθε αναφορά που εξάγεται από τον \lib{CorefAnnotator} προστίθεται στις λίστες \python{coref_clusters} των \python{Doc} και \python{Token} αν δεν υπάρχει ήδη.

\subsection{Ανάθεση σημασιολογικών ρόλων}\label{subsec:4-allenSRL}
Για τη διαδικασία της ανάθεσης σημασιολογικών ρόλων χρησιμοποιείται το αντίστοιχο μοντέλο~\cite{he2017deep} της βιβλιοθήκης \libcite[Python!]{AllenNLP}.
Εδώ, η βιβλιοθήκη χρησιμοποιεί ως βάση το \lib{spaCy} για τον κατακερματισμό σε μονάδες και την επισημείωση μερών του λόγου.
Κάθε λεκτική μονάδα που είναι ρήμα χρησιμοποιείται ως κατηγόρημα και η υπόλοιπη δομή εξάγεται μέσω του εκπαιδευμένου μοντέλου.
Αυτό αποτελεί έναν σημαντικό περιορισμό καθώς δε γίνεται σωστή ανίχνευση βοηθητικών ρημάτων.
Στην \fref{lst:srl} παρουσιάζεται μια ενδεικτική έξοδος του μοντέλου καθώς και ένα πιθανό πρόβλημα με τα βοηθητικά ρήματα.

\begin{listing}
    \begin{minted}{python}
>>> from allennlp.predictors.predictor import Predictor
>>> predictor = Predictor.from_path(
...     "https://s3-us-west-2.amazonaws.com/allennlp/"
...     "models/srl-model-2018.05.25.tar.gz"
... )
>>> [
...     verb["description"]
...     for verb in predictor.predict(
...         "The keys, which were needed to access the building, were locked in the car."
...     )["verbs"]
... ]
['The keys , which [V: were] needed to access the building , were locked in the car .',
 '[ARG1: The keys] , [R-ARG1: which] were [V: needed] [ARGM-PRP: to access the building] , were locked in the car .',
 'The keys , which were needed to [V: access] [ARG1: the building] , were locked in the car .',
 'The keys , which were needed to access the building , [V: were] locked in the car .',
 '[ARG1: The keys , which were needed to access the building ,] were [V: locked] [ARGM-LOC: in the car] .']
\end{minted}
    \lcaption{Παράδειγμα εξόδου SRL του μοντέλου του \lib{AllenNLP}}{%
        Παρατηρούμε ότι για τα δύο βοηθητικά ρήματα \engquote{were} δημιουργείται ξεχωριστή καταχώρηση
        αντί να συγχωνεύονται με τις εκφράσεις \engquote{were needed} και \engquote{were locked}.%
    }\label{lst:srl}
\end{listing}

Καθώς πρόκειται για μοντέλο με αρκετές παραμέτρους, η φόρτωσή του στη μνήμη και η επαγωγή του αποτελέσματος έχει σημαντικό υπολογιστικό κόστος.
Για αυτό τον λόγο, χρησιμοποιείται ξεχωριστή διεργασία υπεύθυνη αποκλειστικά για αυτές τις διαδικασίες.
Το υπόλοιπο πρόγραμμα επικοινωνεί με τη διαδικασία μέσω δύο ουρών (\python{Queue} από \python{multiprocessing}).
Η μία χρησιμοποιείται για να στείλει την πρόταση στην οποία θα γίνει η ανάλυση, χωρίς να χρειάζεται να περιμένει ο αποστολέας και από τη δεύτερη λαμβάνεται η έξοδος του μοντέλου όταν αυτή είναι έτοιμη.
Σημειώνεται ότι απαιτείται \newtermprint[Process]{διεργασία} και όχι απλό \newtermprint[Thread]{νήμα}
λόγω του \newterm[Python]{Καθολικ\rr{ού}{ό} \rr{κλειδώματος}{Κλείδωμα} Διερμηνέα}{Global Interpreter Lock - GIL} της \en{Python} που επιτρέπει την εκτέλεση ενός και μόνου νήματος σε κάθε χρονική στιγμή.
Δηλαδή, δεν υπάρχουν πλεονεκτήματα στον χρόνο εκτέλεσης λόγω παραλληλισμού εκτός και αν χρησιμοποιηθεί ξεχωριστή διεργασία.

Η διαδικασία της ανάθεσης σημασιολογικών ρόλων χρησιμοποιείται για τη δημιουργία υποπροτάσεων (\fref{subsec:4-subsentences}) και το σκεπτικό αυτό αιτιολογείται στη συνέχεια.

\subsection{Υποπροτάσεις}\label{subsec:4-subsentences}
Χρησιμοποιείται ο όρος \newtermprint[Subsentence]{Υποπρόταση} για να περιγράψει ένα οποιοδήποτε υποσύνολο λεκτικών μονάδων μιας πρότασης.
Σκοπός των υποπροτάσεων στο \projectname{} είναι η ανεύρεση πολλαπλών προθέσεων του χρήστη σε κάθε πρόταση (βλέπε και \fref{subsec:multi-intent}).
Αυτό υλοποιείται με την εκμετάλλευση των αποτελεσμάτων της ανάλυσης σημασιολογικών ρόλων.

\subsubsection{Δημιουργία}\label{subsec:4-subsentence-create}
Κύρια ιδέα της υλοποίησης αποτελεί η αντιστοίχηση κάθε \newterm[Ανάθεση Σημασιολογικών Ρόλων - Semantic Role Labeling - SRL]{Δομή\dd{ς} Κατηγορήματος-Ορισμάτων}{Predicate-Argument Structure} με μια πιθανή πρόθεση του χρήστη.
Σε αντίθεση με τον διαχωρισμό σε συνεχόμενα υποσύνολα της πρότασης ή απλές προτάσεις (\en{Clause}),
η προσέγγιση αυτή επιτρέπει τη σωστή ανάλυση ακόμη και πιο δύσκολων συντακτικών δομών όπως της \newterm{Ανύψωση\dd{ς} Δεξιού Κόμβου}{Right Node Raising} (βλέπε και \fref{subsec:linguistics}).
Η διαδικασία που ακολουθείται είναι η εξής:
\begin{compactenum}
    \item Για κάθε ρήμα στην έξοδο του μοντέλου σημασιολογικών ρόλων (\fref{subsec:4-allenSRL}), δημιουργείται ένα αντικείμενο τύπου \python{SubSentence}
    \item Αποθηκεύονται οι θέσεις των ορισμάτων (\ARGs{}) και των τροποποιητών (\ARGM{})
    \item Γίνεται συσχετισμός των υποπροτάσεων μεταξύ τους
          \begin{compactenum}
              \item Υποπροτάσεις στις οποίες τα υποσύνολα των κατηγορημάτων και ορισμάτων δεν έχουν αλληλεπικάλυψη θεωρούνται \enquote{συμβατές}:
                    \begin{equation}
                        (\{ARG\}_1 \cup \{ARGM\}_1 ) \cap (\{ARG\}_2 \cup \{ARGM\}_2) = \varnothing
                    \end{equation}

                    Αυτή η διαδικασία καθίσταται απαραίτητη λόγω των αποτελεσμάτων της ανάθεσης σημασιολογικών ρόλων που μερικές φορές μπορεί να παρουσιάζει αντιφάσεις.
              \item Μια υποπρόταση θεωρείται ότι \enquote{τροποποιεί} μιαν άλλη αν το ρήμα της πρώτης βρίσκεται μέσα στο διάστημα των τροποποιητών (\ARGM{}) της δεύτερης.
                    Αυτές οι υποπροτάσεις θεωρούνται συμβατές εκτός και αν υπάρχει ο πρώτος τύπος αλληλεπικάλυψης.
          \end{compactenum}
\end{compactenum}

Για παράδειγμα, στην πρόταση \engquote{Open your left hand and then extend it while saying hello} δημιουργούνται τρεις υποπροτάσεις, όλες συμβατές μεταξύ τους,
από την έξοδο της ανάθεσης σημασιολογικών ρόλων:
\begin{compactenum}
    \item \en{[V: Open] [ARG1: your left hand] and then extend it while saying hello.}
    \item \en{Open your left hand and [ARGM-TMP: then] [V: extend] [ARG1: it] [ARGM-TMP: while saying hello].}
    \item \en{Open your left hand and then extend it while [V: saying] [ARG1: hello].}
\end{compactenum}
Όπως είναι φανερό, καθεμία από αυτές αντιστοιχεί σε μια ξεχωριστή πρόθεση χρήστη.

Τέτοιες προσεγγίσεις δεν συναντώνται στη βιβλιογραφία --- τουλάχιστον σε αυτές που βρέθηκαν κατά τη διάρκεια της έρευνας για την παρούσα διπλωματική.
Οι πιο σχετικές μέθοδοι που συναντήθηκαν είναι αυτές που αντιστοιχίζουν κάθε δομή κατηγορήματος-ορίσματος σε κάποια πρόθεση του συστήματος~\cite{tur2005semi,hakkani2015clustering}
(γίνεται αναφορά και στην \fref{subsec:intents-and-entities})
και στο~\cite{diamantopoulos2017software} όπου χρησιμοποιείται η ανάθεση σημασιολογικών ρόλων για την ανάλυση λειτουργικών απαιτήσεων λογισμικού και η αντιστοίχησή τους σε μια οντολογία.

\subsubsection{Συνδυασμός}\label{subsec:4-subsentence-combine}
Μετά τη διαδικασία δημιουργίας τους, οι υποπροτάσεις συνδυάζονται σε δομές.
Κάθε συνδυασμός αποτελεί μια υπόθεση του διαχωρισμού της πρότασης σε προθέσεις και οντότητες.
Ακολουθείται η εξής διαδικασία:
\begin{compactenum}
    \item Η δημιουργία των συνδυασμών γίνεται στο δυναμοσύνολο\footnote{\newtermc{Δυναμοσύνολο}{Power Set} $P(X)$ ενός συνόλου $X$ είναι το σύνολο όλων των υποσυνόλων του $P(X) = \{A: A \subseteq X\}$}
          όλων των υποπροτάσεων που δημιουργήθηκαν
    \item Επιλέγεται ένας συνδυασμός με τα εξής κριτήρια:
          \begin{compactenum}
              \item αρχίζοντας από αυτούς που περιέχουν μεγαλύτερο αριθμό υποπροτάσεων
              \item αν δεν έχουν χρησιμοποιηθεί όλες οι υποπροτάσεις σε κάποιον προηγούμενο συνδυασμό,
              \item αν όλες οι υποπροτάσεις του συνδυασμού είναι συμβατές μεταξύ τους
          \end{compactenum}
    \item Η διαδικασία σταματάει όταν έχουν χρησιμοποιηθεί όλες οι υποπροτάσεις σε κάποιον συνδυασμό
\end{compactenum}

\subsection{Επιλογή κειμένου υποπρότασης}\label{subsec:4-powerset}
Στόχος αυτού του βήματος είναι η επιλογή του τελικού κειμένου μιας υποπρότασης σύμφωνα με την έξοδο της μονάδας NLU.
Οι πιθανές επιλογές γίνονται στα δυναμοσύνολα συνδυασμών τροποποιητών και συναναφορών.
Επιλέγεται αυτή που παρουσιάζει τη μεγαλύτερη αξιολόγηση στην έξοδο της μονάδας NLU.

\subsubsection{Τροποποιητές}
Κύριο κορμό του κειμένου, που περιλαμβάνεται σε κάθε εκδοχή του, αποτελεί το κατηγόρημα και τα ορίσματά του \ARGs{}.
Οι τροποποιητές \ARGM{} που θα περιληφθούν σε κάθε έκδοση του κειμένου επιλέγονται μέσα από το δυναμοσύνολο όλων των τροποποιητών που δεν ανήκουν σε κάποια άλλη υποπρόταση του ίδιου συνδυασμού με την τρέχουσα υποπρόταση.
Για παράδειγμα, στην υποπρόταση που δημιουργήθηκε από την εξής δομή κατηγορήματος-ορισμάτων
\engquote{Open your left hand and [ARGM-TMP: then] [V: move] [ARGM-DIR: forwards] [ARGM-TMP: while saying hello]}
το βασικό κείμενο είναι μόνο το ρήμα \engquote{[V: move]}
και το σύνολο των τροποποιητών είναι:
\begin{compactenum}
    \item \en{[ARGM-TMP: then]}
    \item \en{[ARGM-DIR: forwards]}
    \item \en{[ARGM-TMP: while saying hello]}
\end{compactenum}
Ο τελευταίος όμως θα ανήκει σε ξεχωριστή υποπρόταση του ίδιου συνδυασμού καθώς περιλαμβάνει το ρήμα \engquote{saying}
οπότε οι πιθανές επιλογές που θα αξιολογηθούν είναι:\\
\begin{tabulary}{\linewidth}{L @{ --- } L}
    \en{then move forwards} & επιλέγονται όλοι οι τροποποιητές    \\
    \en{then move}          & επιλέγεται μόνο ο πρώτος            \\
    \en{move forwards}      & επιλέγεται μόνο ο δεύτερος          \\
    \en{move}               & δεν επιλέγεται κανένας τροποποιητής
\end{tabulary}

\subsubsection{Συναναφορές}
Επιπλέον, για καθεμία από τις προηγούμενες επιλογές, επιλέγονται και οι συναναφορές οι οποίες θα αντικατασταθούν με το κείμενο της κύριας αναφοράς τους.
Για παράδειγμα, στην υποπρόταση
\engquote{Open your left hand and [ARGM-TMP: then] [V: extend] [ARG1: it] [ARGM-TMP: while saying hello].}
ο κύριος κορμός είναι τα \engquote{[V: extend] [ARG1: it]} όπου το \engquote{it} αναφέρεται στο \engquote{your left hand}.
Συμπεριλαμβανόμενης της διαδικασίας επιλογής τροποποιητών, όλες οι πιθανές επιλογές του τελικού κειμένου της υποπρότασης προκύπτουν:
\begin{compactenum}
    \item{then extend it}
    \item{then extend your left hand}
    \item{extend it}
    \item{extend your left hand}
\end{compactenum}

\subsection{Κατανόηση φυσικής γλώσσας}\label{subsec:4-NLU}
Η διαδικασία κατανόησης φυσικής γλώσσας γίνεται σε επίπεδο υποπρότασης καθώς δε χρησιμοποιούνται μοντέλα μηχανικής εκμάθησης που να επιτρέπουν την ταξινόμηση πολλών ετικετών (\en{Multi-Label Classification}).
Η αναγνώριση της πρόθεσης του χρήστη και η εξαγωγή οντοτήτων γίνεται πάνω σε καθεμία από τις επιλογές τελικού κειμένου που αναφέρθηκαν προηγουμένως (\fref{subsec:4-powerset}).
Για κάθε υποπρόταση, επιλέγεται το κείμενο που παρουσιάζει μεγαλύτερη βαθμολογία στην έξοδο της μονάδας NLU.

Κατά την ανάλυση του κειμένου πραγματοποιείται επίσης μια διαδικασία \newterm{Λημματοποίηση\dd{ς}}{Lemmatization} των ρημάτων.
Η λημματοποίηση αποτελεί τη διαδικασία της αντιστοίχισης κάθε επιμέρους γλωσσικού τύπου με το λήμμα στο οποίο ανήκει.
Αυτό στοχεύει στη βελτίωση της απόδοσης σε προτάσεις που χρησιμοποιούν ρήματα σε κλίσεις που δεν συναντώνται στα δεδομένα εκπαίδευσης.
Η διαδικασία γίνεται με τη βοήθεια του \libcite[Python!]{spaCy}.

\subsubsection{Snips}\label{subsec:4-snips}
Η βιβλιοθήκη \libcite[Python!]{Snips} που χρησιμοποιήθηκε ως βασικό μοντέλο πραγματοποιεί την αναγνώριση προθέσεων και την πλήρωση των σημασιολογικών υποδοχέων.
Το μοντέλο της βιβλιοθήκης αρχικά εκπαιδεύεται στο σύνολο δεδομένων.

Ένα από τα πλεονεκτήματα της βιβλιοθήκης σε σχέση με αντίστοιχες είναι η υποστήριξη \newterm{Αυτόματα Επεκτάσιμων Οντοτήτων}{Automatically Extensible Entities}\footnote{%
    \breakurl{https://snips-nlu.readthedocs.io/en/latest/data_model.html\#automatically-extensible-entities}%
}.
Αυτές μπορούν να χρησιμοποιηθούν ώστε συγκεκριμένες οντότητες να μπορούν να πάρουν τιμές που δεν συναντώνται στα δεδομένα εκπαίδευσης.
Για παράδειγμα, στην πρόθεση \intentr{Talk}, η οντότητα του κειμένου \entity{Text} θα μπορούσε να πάρει οποιαδήποτε τιμή χρησιμοποιήσει ο χρήστης.

Παρά τη δυνατότητα των επεκτάσιμων οντοτήτων, μερικές φορές είναι επιθυμητή η ύπαρξη κάποιας ελευθερίας στον προσδιορισμό των οντοτήτων αλλά και η ταυτόχρονη ύπαρξη περιορισμών.
Για παράδειγμα, στην πρόθεση \intentr{ArmMotion}, η οντότητα \entity{armMotion} (ο τύπος της κίνησης) μπορεί να πάρει τις προκαθορισμένες τιμές \entity{OFFER}, \entity{UP}, \entity{WAVE}, \ldots{} αλλά πρέπει να υποστηρίζει και πιο γενικές εκφράσεις και συνώνυμα.
Έτσι δημιουργούνται διαφορετικές προθέσεις για κάθε πιθανή τιμή της \entity{armMotion}:
\begin{compactenum}
    \item \intent{ArmMotionOffer}
    \item \intent{ArmMotionUp}
    \item \intent{ArmMotionWave}
    \item \intent{ArmMotionExtend}
    \item \intent{ArmMotionOpen}
    \item \intent{ArmMotionClose}
\end{compactenum}
οι οποίες, όταν ανιχνεύονται, μεταφράζονται στην πρόθεση \intentr{ArmMotion} με την αντίστοιχη τιμή της οντότητας \entity{armMotion}.

Επιπλέον, η βιβλιοθήκη υποστηρίζει την εξαγωγή οντοτήτων σε κείμενο ξεχωριστά από τη διαδικασία αναγνώρισης πρόθεσης.
Οπότε, μπορεί να χρησιμοποιηθεί κάποιο μοντέλο ταξινόμησης ακολουθιών για την πρόθεση και μετά, ανεξάρτητα, να χρησιμοποιηθεί η δυνατότητα εξαγωγής οντοτήτων του \lib{Snips} σύμφωνα με την αναγνωρισμένη πρόθεση.

\begin{code}
    \begin{minted}{python}
>>> from snips_nlu import SnipsNLUEngine
>>> nlu = SnipsNLUEngine.from_path("/path/to/dir")
>>> nlu("move forwards")
{'input': 'move forwards',
 'intent': {'intentName': 'BodyMotionForwards', 'probability': 1.0},
 'slots': []}
>>> nlu("raise your left hand")
{'input': 'raise your left hand',
 'intent': {'intentName': 'ArmMotionUp', 'probability': 0.8186328266611467},
 'slots': [{'range': {'start': 11, 'end': 15},
   'rawValue': 'left',
   'value': {'kind': 'Custom', 'value': 'LEFT'},
   'entity': 'Arm',
   'slotName': 'arm'}]}
    \end{minted}
    \caption{Παράδειγμα χρήσης εκπαιδευμένου μοντέλου \lib{Snips}}

    \begin{minted}{python}
>>> from r4a_nao_nlp.engines import shared
>> shared.init()
>>> shared.parse("move forwards")
SnipsResult(score=1.0, name='BodyMotion', slots=(SnipsSlot(range=range(-1, -1), value='FORWARDS', entity='Direction', name='direction'),))
>>> shared.parse("raise your left hand")
SnipsResult(score=0.8186328266611467, name='ArmMotion', slots=(SnipsSlot(range=range(-1, -1), value='UP', entity='ArmMotionType', name='armMotion'), SnipsSlot(range=range(11, 15), value='LEFT', entity='Arm', name='arm')))
    \end{minted}
    \caption{Παράδειγμα χρήσης του μοντέλου μαζί με το σύστημα μετάφρασης και τη χρήση βοηθητικών αντικειμένων \python{SnipsResult} και \python{SnipsSlot}}
\end{code}

\subsubsection{Αξιολόγηση συνδυασμών}\label{subsec:4-score}
Αφού επιλεχθούν τα τελικά κείμενα όλων των υποπροτάσεών τους, οι συνδυασμοί αξιολογούνται σύμφωνα με τις τελικές τους αξιολογήσεις.
Χρησιμοποιείται απλώς ο μέσος όρος όλων των αξιολογήσεων των υποπροτάσεων για την τελική αξιολόγηση.
Άλλες επιλογές θα αποτελούσαν η ελάχιστη τιμή όλων των αξιολογήσεων ή το γινόμενό τους.
Από όλους τους συνδυασμούς που έχουν δημιουργηθεί, επιλέγεται αυτός με τη μεγαλύτερη αξιολόγηση.

\subsection{Δημιουργία κατευθυνόμενου γράφου}\label{subsec:4-DiGraph}
Σε αυτό το βήμα, το \projectname{} δημιουργεί έναν κατευθυνόμενο γράφο που αναπαριστά τον επιλεγμένο συνδυασμό.
Αυτός επεκτείνεται με κάθε πρόταση του κειμένου και μπορεί να χρησιμοποιηθεί για την προγραμματιστική πρόσβαση στις υποπροτάσεις, τις μεταξύ τους σχέσεις και τις προθέσεις και οντότητες που έχουν ανιχνευθεί στο κείμενο εισόδου.

Για τη διαδικασία αυτή, επιλέχθηκε το \libcite[Python!]{networkx} λόγω της απλότητας στη χρήση, την υποστήριξη δημιουργίας γραφημάτων από γράφους και τις καλές του επιδόσεις.

Στον γράφο, κάθε υποπρόταση αποτελεί έναν κόμβο.
Οι λεκτικές μονάδες που δεν χρησιμοποιούνται σε καμία υποπρόταση χρησιμοποιούνται ως ετικέτες των ακμών που συνδέουν διαδοχικές υποπροτάσεις.
Η κατεύθυνση των υποπροτάσεων καθορίζεται από τη θέση του ρήματος στο αρχικό κείμενο.

Οι υποπροτάσεις που λειτουργούν ως τροποποιητές ενώνονται με τις υποπροτάσεις που τροποποιούν
και οι ακμές που τις ενώνουν έχουν ως ετικέτα τους τροποποιητές \ARGM{} που δεν έχουν χρησιμοποιηθεί κατά τη διαδικασία ανάλυσης προθέσεων.
Για παράδειγμα, αν έχουμε τις εξής υποπροτάσεις:
\begin{compactenum}
    \item \en{[V: Extend] [ARG1: both of your hands] [ARGM-TMP: while moving forwards]}
    \item \en{Extend both of your hands while [V: moving] [ARGM-DIR: forwards]}
\end{compactenum}
Η δεύτερη τροποποιεί την πρώτη.
Η ακμή που τις ενώνει θα έχει ως ετικέτα τη λέξη \engquote{while} καθώς βρίσκεται στους τροποποιητές της δομής κατηγορήματος-ορισμάτων του \engquote{extend} αλλά όχι στα ορίσματα του \engquote{moving}.

\subsubsection{Μορφοποίηση εξόδου}\label{subsec:4-format}
Πέρα από την απευθείας χρήση του αντικείμενου του γράφου μπορούν να χρησιμοποιηθούν άλλοι τρόπου μορφοποίησης της εξόδου οι οποίοι μπορούν να προκύψουν από τη δομή του γράφου.
Μια δομημένη λεκτική περιγραφή μπορεί να αξιοποιηθεί από ένα άλλο σύστημα για την αυτόματη δημιουργία του κώδικα της ρομποτικής εφαρμογής.

% vim:ts=4:sw=4:expandtab:fo-=tc:tw=120

\chapter{Πειράματα \& Αποτελέσματα}\label{chap:results}% TODO: write blabla to fit graphs
\section{Πιθανά προβλήματα}\label{sec:problems}
\subsection{Λόγω \lib{spaCy}}\label{subsec:problems-spacy}
Καθώς τα μοντέλα της αγγλικής γλώσσας που παρέχονται από τη βιβλιοθήκη \lib{spaCy} χρησιμοποιούνται και από την υλοποίηση SRL της βιβλιοθήκης \lib{AllenNLP},
σφάλματα στην πρώτη μπορούν να οδηγήσουν σε λάθος αποτελέσματα στη δεύτερη.
Για παράδειγμα, στις προτάσεις της καταχώρησης \ref{lst:problem-POS} υπάρχουν λάθη στην \newterm{Επισημείωση Μερών του Λόγου}{Part-of-Speech Tagging}
και η εσφαλμένη ανάθεση σημασιολογικών ρόλων φαίνεται στην \fref{lst:problem-POS-SRL}.

Μια λύση για αυτό το πρόβλημα αποτελεί η επανεκπαίδευση των μοντέλων της \lib{spaCy} πάνω σε νέα δεδομένα\footnote{\url{https://explosion.ai/blog/pseudo-rehearsal-catastrophic-forgetting}}.

\begin{listing}[p] % Using p here since they fit perfectly in a page
    \begin{minted}{python}
>>> import spacy
>>> nlp = spacy.load("en_core_web_sm")  # Μοντέλο που χρησιμοποιεί ως προεπιλογή η AllenNLP
>>> # Σημειώνεται ότι η πρόσφατη νεώτερη έκδοση δεν εμφανίζει όλα τα προβλήματα που αναφέρονται εδώ
>>> nlp._meta["version"]
'2.0.0'
>>> doc = nlp("Replay the motion named XXX")
>>> doc[0].pos_  # Σωστό
'VERB'
>>> doc = nlp("Replay motion named XXX")  # Χωρίς το "the"
>>> doc[0].pos_  # Θα έπρεπε και πάλι να είναι "VERB"
'NOUN'
>>> # Προβλήματα με κατευθύνσεις "left" και "right"
>>> doc = nlp("Turn right")
>>> doc[-1].pos_  # Σωστό
'ADV'
>>> doc = nlp("Turn right and then left")
>>> doc[-1].pos_  # Θα έπρεπε και πάλι να είναι "ADV"
'VERB'
    \end{minted}
    \caption{Σφάλματα στην αντιστοίχιση ετικετών μερών του λόγου}\label{lst:problem-POS}
\end{listing}
\begin{listing}[p]
    \begin{minted}{python}
>>> from allennlp.predictors.predictor import Predictor
>>> predictor = Predictor.from_path(
... "https://s3-us-west-2.amazonaws.com/allennlp/"
... "models/srl-model-2018.05.25.tar.gz"
... )
>>> predictor = Predictor.from_path(
...     "https://s3-us-west-2.amazonaws.com/"
...     "allennlp/models/srl-model-2018.05.25.tar.gz"
... )
>>> [verb["description"] for verb in predictor.predict("Turn right")["verbs"]]
['[V: Turn] [ARGM-DIR: right]']
>>> [verb["description"] for verb in predictor.predict("Turn right and then left")["verbs"]]
['[V: Turn] [ARGM-DIR: right] and then left',
 'Turn right and [ARGM-TMP: then] [V: left]']
    \end{minted}
    \caption{Πως επηρεάζεται η υλοποίηση SRL από τα σφάλματα στην αντιστοίχιση ετικετών μερών του λόγου}\label{lst:problem-POS-SRL}
\end{listing}

\subsection{Λόγω ανάθεσης σημασιολογικών ρόλων}
Ένα παράδειγμα προβλήματος που μπορεί να προκύψει από την ανάθεση σημασιολογικών ρόλων είχε παρουσιαστεί στην \fref{lst:srl}.
Ένα δεύτερο παράδειγμα παρατίθεται στην \fref{lst:srl-problem}.

\begin{listing}[t]
    \begin{minted}{python}
>>> from allennlp.predictors.predictor import Predictor
>>> predictor = Predictor.from_path(
...     "https://s3-us-west-2.amazonaws.com/allennlp/"
...     "models/srl-model-2018.05.25.tar.gz"
... )
>>> [
...     verb["description"]
...     for verb in predictor.predict("Stand up and move forwards 5 meters")["verbs"]
... ]
['[V: Stand] up and move forwards 5 meters',
 'Stand up and [V: move] [ARGM-DIR: forwards] [ARG2: 5 meters]']
\end{minted}
    \lcaption{Σφάλμα στην ανάθεση σημασιολογικών ρόλων}{%
        Το \engquote{up} δεν συμμετέχει στη πρώτη δομή.%
    }\label{lst:srl-problem}
\end{listing}

\subsection{Λόγω επισήμανσης συναναφορών}
Όπως αναφέρθηκε, το πρόβλημα της επισήμανσης συναναφορών (\fref{subsec:coref}) θεωρείται ακόμα ανοιχτό στον χώρο της κατανόησης φυσικής γλώσσας.
Για αυτό, συναντήθηκαν διάφορες προτάσεις στις οποίες κανένα από τα μοντέλα δεν κατάφερε να βρει επιτυχώς όλες τις συναναφορές που υπήρχαν.
Σε τέτοιες προτάσεις, η μηχανή της snips μπορεί να αποτύχει να βρει όλες τις οντότητες ή ακόμα και τη σωστή πρόθεση του χρήστη.

Επίσης προβλήματα μπορούν να δημιουργηθούν όταν γίνει λάθος ομαδοποίηση των συναναφορών και μια δευτερεύουσα αναφορά αντικατασταθεί με μια πρωτεύουσα αναφορά που ανήκει σε διαφορετική οντότητα.

\subsection{Λόγω NLU}
Προβλήματα μπορούν να προκύψουν όταν το εκπαιδευμένο μοντέλο \NLU{} δεν μπορεί να βρει την πρόθεση του χρήστη στην τελική πρόταση μετά τις υπόλοιπες μετατροπές.
Συνήθως, απαιτείται η ενίσχυση της βάσης δεδομένων με περισσότερα παραδείγματα.

\section{Αποτελέσματα σε σενάρια χρήσης}
Η δοκιμή του συστήματος γίνεται με διάφορα σενάρια χρήσης.
Η έξοδος του συστήματος μορφοποιείται σε γράφο με τη βοήθεια του \libcite{Graphviz} και παρουσιάζεται στα σχήματα αυτής της ενότητας.

\subsection{Δοκιμαστικά σενάρια χρήσης}
Σε αυτή την υποενότητα παρατίθενται τα σενάρια που συγγράφηκαν κατά τη διάρκεια της υλοποίησης της διπλωματικής.
Αποτελούν παραδείγματα για τα οποία υπάρχει η προσδοκία καλής απόδοσης.
Λαμβάνουν υπόψη τους περιορισμούς του συστήματος και χρησιμοποιούνται για να επιδείξουν τις δυνατότητές του συνδυάζοντας αρκετές από τις έννοιες που έχουν αναφερθεί προηγουμένως.

\begin{figure}
    \makeatletter%
    \centering
    \def\svgscale{0.85}%
    \ig@escapeunderscore{\input{images/graphs/with-modifier.pdf_tex}}
    \lcaption{Παράδειγμα από δοκιμαστικό σενάριο χρήσης}{%
        Αρχικό κείμενο: \engquote{Recognize speech without moving}.

        Σε αυτή την πρόταση το σύστημα αναγνωρίζει την πρόθεση \intent{BodyMotion} χωρίς ορίσματα.
        Η ακινησία (\engquote{without moving}) δεν αποτελεί επιλογή του \metamodel{} σε αυτή την ενέργεια καθώς είναι η υπόρρητη προεπιλογή.
        Έτσι, ένα σύστημα που αξιοποιεί την έξοδο του \projectname{} πρέπει να αφαιρέσει τον πλεονασμό.
    }%
    \def\svgscale{0.85}
    \ig@escapeunderscore{\input{images/graphs/multi-entity-double-intent.pdf_tex}}
    \lcaption{Παράδειγμα από δοκιμαστικό σενάριο χρήσης}{%
        Αρχικό κείμενο: \engquote{Enable the leds of your chest and legs and go left}.

        Εδώ φαίνεται πως το σύστημα έχει τη δυνατότητα να ανιχνεύει πολλαπλές οντότητες σε μια πρόθεση αλλά και να διαχωρίζει δύο προθέσεις σε μια πρόταση.
    }%
    \makeatother%
\end{figure}
\begin{figure}
    \centering
    \makeatletter%
    \def\svgscale{0.85}%
    \ig@maxfigure{0.4\textheight}{%
        \ig@escapeunderscore{\input{images/graphs/replay1.pdf_tex}}}{%
        \lcaption{Παράδειγμα από δοκιμαστικό σενάριο χρήσης}{%
            Αρχικό κείμενο: \engquote{Learn motion \engquote{bakasana}. After that, replay it}.
        }}%
    \def\svgscale{0.85}%
    \ig@maxfigure{0.5\textheight}{%
        \ig@escapeunderscore{\input{images/graphs/replay2.pdf_tex}}}{%
        \lcaption{Παράδειγμα από δοκιμαστικό σενάριο χρήσης}{%
            Αρχικό κείμενο: \engquote{Learn motion \engquote{shake hips}. After that, take 5 steps forward, replay the first motion while extending your arms}.

            Εδώ το σύστημα δεν είναι ικανό να αναγνωρίσει αναφορές που βασίζονται σε λογικές διαδικασίες.
            Για την κατανόηση του \engquote{first motion} θα απαιτούνταν η ταξινόμηση των οντοτήτων σε κλάσεις και η κατανόηση της αρίθμησής τους.
            Αυτή η διαδικασία μπορεί να αναληφθεί από λογισμικό που επεξεργάζεται την έξοδο του \projectname{}.
        }}%
    \makeatother%
\end{figure}
\begin{figure}
    \centering
    \makeatletter%
    \def\svgscale{0.85}%
    \ig@maxfigure{0.4\textheight}{%
        \ig@escapeunderscore{\input{images/graphs/coref-hand.pdf_tex}}}{%
        \lcaption{Παράδειγμα από δοκιμαστικό σενάριο χρήσης}{%
            Αρχικό κείμενο: \engquote{Detect touch on your right hand while waving it}.

            Εδώ γίνεται σωστή επίλυση αναφοράς, μεταξύ διαφορετικών προθέσεων, της αντωνυμίας \engquote{it}.
        }}%
    \def\svgscale{0.85}%
    \ig@maxfigure{0.5\textheight}{%
        \ig@escapeunderscore{\input{images/graphs/right-node-raising.pdf_tex}}}{%
        \lcaption{Παράδειγμα από δοκιμαστικό σενάριο χρήσης}{%
            Αρχικό κείμενο: \engquote{Raise and open your left hand without extending it}.

            Σε αυτό το παράδειγμα το σύστημα επιτυχώς διαχωρίζει μια δύσκολη συντακτική δομή που χρησιμοποιεί την ανύψωση δεξιού κόμβου (βλέπε \fref{subsec:linguistics}).
            Η οντότητα \engquote{left arm} χρησιμοποιείται επιτυχώς και στις τρεις προθέσεις κίνησης του χεριού:
            στις δύο πρώτες ως όρισμα του ρήματος στη δομή ανύψωσης δεξιού κόμβου και στην τελευταία μέσω αναφοράς με την αντωνυμία \engquote{it}.
        }}%
    \makeatother%
\end{figure}
\ig[type=inkscape]{long-corefs}{%
    \lcaption{Παράδειγμα από δοκιμαστικό σενάριο χρήσης}{%
        Αρχικό κείμενο: \engquote{Open your left hand and then extend it while saying hello.
            Then turn left and offer it.
            If you detect a human face, turn right.
            Else, move forwards}.

        Σε αυτό το παράδειγμα παρουσιάζεται η δυνατότητα του συστήματος να βρίσκει συναναφορές που βρίσκονται σε μεγάλη απόσταση σε σχέση με τη κύρια αναφορά.
    }%
}
\ig[type=inkscape]{other-coref}{%
    \lcaption{Παράδειγμα από δοκιμαστικό σενάριο χρήσης}{%
        Αρχικό κείμενο: \engquote{Move forwards, open your left hand and turn left.
            If you see a human, close your hand.
            Else, sit down.}.

        Εδώ φαίνεται πως η αναφορά \engquote{your hand} επιλύεται και χρησιμοποιείται σωστά ως όρισμα στη δεύτερη κίνηση του χεριού.
    }%
}

\FloatBarrier
\subsection{Πραγματικά σενάρια χρήσης}
Στη συνέχεια, ελέγχθηκε η αποτελεσματικότητα του συστήματος στην παραγωγή κώδικα από σενάρια που συλλέχθηκαν από
χρήστες\footnote{Τα σενάρια που υποβλήθηκαν βρίσκονται στον σύνδεσμο \url{https://goo.gl/V5FyDH}} που δε συμμετείχαν στη συγγραφή αυτής της εργασίας.

Λόγω της δυσκολίας συλλογής μεγάλου πλήθους δεδομένων εκπαίδευσης (που αποτελεί χρονοβόρα διαδικασία η οποία θεωρήθηκε εκτός των στόχων αυτής της εργασίας),
το \projectname{} παρουσιάζει μειωμένη απόδοση σε εισόδους που είναι γραμμένες με αρκετά διαφορετικό τρόπο.
Για παράδειγμα, εμφανίζονται προβλήματα στην αναγνώριση της πρόθεσης προτάσεων που είναι γραμμένες στο τρίτο πρόσωπο (\engquote{NAO does X}) αντί για προστακτική (\engquote{Do X}).

Επίσης, όπου κρίνεται απαραίτητο, γίνονται γραμματικές, ορθογραφικές και συντακτικές διορθώσεις και τοποθετούνται εισαγωγικά.
Σε αυτή την υποενότητα παρουσιάζονται ενδεικτικά μερικά αποτελέσματα του συστήματος σε αυτά τα κείμενα εισόδου.

\ig[type=inkscape]{22}{%
    \lcaption{Παράδειγμα από πραγματικό σενάριο χρήσης}{%
        Αρχικό κείμενο: \engquote{If you see me leaving the house warn me about potential bad weather}.

        Η πρόταση αυτή είναι αρκετά γενική και περιλαμβάνει οντότητες που το σύστημα δεν αναγνωρίζει.
        Για τη σωστή ανάλυσή της θα απαιτούνταν κάποιο σύστημα υψηλότερης λογικής που θα μπορούσε να κατασκευάσει έναν αλγόριθμο από γενικές έννοιες.
    }%
}
\ig[type=inkscape]{0}{%
    \lcaption{Παράδειγμα από πραγματικό σενάριο χρήσης}{%
        Αρχικό κείμενο: \engquote{Initially, say \engquote{Hello everyone}.
            Then say the current date and time.
            At the same time, if someone touches you, turn the leds on.
            Finally, say \engquote{Nice to meet you my friend}.
        }
    }%
}
\ig[svgwidth=\linewidth,type=inkscape]{1}{%
    \lcaption{Παράδειγμα από πραγματικό σενάριο χρήσης}{%
        Αρχικό κείμενο: \engquote{Do the following actions forever:
            Initially, look right and then look left.
            At the same time, if someone says \engquote{NAO}, walk one step forward}.
    }%
}
\ig[type=inkscape]{2}{%
    \lcaption{Παράδειγμα από πραγματικό σενάριο χρήσης}{%
        Αρχικό κείμενο: \engquote{First of all, say \engquote{What do you want to ask me?}.
            Then, wait until you hear someone talking.
            If you hear someone talking, say \engquote{Please speak louder, thank you}}.
    }%
}
\ig[svgscale=0.8,type=inkscape]{3}{%
    \lcaption{Παράδειγμα από πραγματικό σενάριο χρήσης}{%
        Αρχικό κείμενο: \engquote{Look for humans, if you detect someone, whisper \engquote{I'm coming}.
            Then turn on all the leds and wait 5 seconds.
            After that, shout \engquote{Your time is due}.
            When this happens, turn the leds off}.
    }%

    Εδώ στις εκφράσεις \engquote{whisper} και \engquote{shout} αποδίδεται ιδιαίτερη σημασία μέσω των βοηθητικών προθέσεων \intent{Shout} και \intent{Whisper} που μεταφράζονται στην υπαρκτή πρόθεση \intent{Talk}.
}
\ig[svgscale=0.8,type=inkscape]{4}{%
    \lcaption{Παράδειγμα από πραγματικό σενάριο χρήσης}{%
        Αρχικό κείμενο: \engquote{Look for humans, if you detect someone, make 3 steps towards that way.
            When this happens, turn on the leds and wait 4 seconds.
            At the same time shout \engquote{Hello human}.
            When this happens, turn the leds off.
            Finally, whisper \engquote{You're doomed}}.
    }%
}
\ig[svgscale=0.8,type=inkscape]{19}{%
    \lcaption{Παράδειγμα από πραγματικό σενάριο χρήσης}{%
        Αρχικό κείμενο: \engquote{Start trying to find a human.
            Once you find one, do the following: Say \engquote{Hello mortal, let's play a luck-based game}.
            Then, say \engquote{Teach me one motion for my left hand}.
            Then, learn the 1st motion.
            Then, say \engquote{Teach me a motion for my right hand}.
            Then, learn the 2nd motion.
            Then, say \engquote{Teach me a motion for my head}'.
            Then, learn the 3rd motion.
            Once you have learned all the motions, say \engquote{Ok now guess which motion I will make}' and start listening for either \engquote{1}, \engquote{2} or \engquote{3}.
            Once you hear \engquote{1}, \engquote{2} or \engquote{3},
            make randomly one of the 3 motions you learned and if that motion is the same as the one the user said say \engquote{You won!},
            else say \engquote{You lost, I told you, you are mortal after all!}}

        Στο τέλος, η σωστή ανίχνευση των προθέσεων αποτυχαίνει καθώς απαιτείται η χρήση λογικής.
    }%
}

% vim:ts=4:sw=4:expandtab:fo-=tc:tw=120

\chapter{Συμπεράσματα \& Μελλοντική Εργασία}\label{chap:conclusions_future}

\section{Συμπεράσματα}
Σε αυτή τη διπλωματική εργασία παρουσιάστηκε η σωλήνωση λογισμικού \projectname{} που υλοποιεί ένα σύστημα επεξεργασίας και κατανόησης φυσικής γλώσσας.
Στόχος του συστήματος είναι η αναγνώριση και διασύνδεση των ενεργειών και οντοτήτων του μέτα-μοντέλου ενεργειών ρομποτικής πλατφόρμας \metamodel{}.
Το σύστημα εκμεταλλεύεται σχέσεις συναναφοράς που υπάρχουν σε όλο το σώμα του κειμένου εισόδου για να επιλύσει αντωνυμίες όπως το \engquote{it}.
Επίσης, χρησιμοποιεί τους σημασιολογικούς ρόλους για την αναγνώριση πολλών προθέσεων (ενεργειών) ανά πρόταση και τη διασύνδεσή τους μέσω τροποποιητών.

Το σύστημα δεν αναλαμβάνει τη διαδικασία αυτόματης δημιουργίας του κώδικα της ρομποτικής εφαρμογής.
Έτσι, το έργο της παρούσας διπλωματικής δεν είναι άμεσα εκμεταλλεύσιμο από τον τελικό χρήστη,
ωστόσο θεωρείται ότι μπορεί να υλοποιηθεί ένα άλλο σύστημα, ανεξάρτητο από το πλαίσιο αυτής της διπλωματικής, που θα μπορεί να αναλάβει την παραγωγή κώδικα εκμεταλλευόμενο τις πληροφορίες που περιλαμβάνονται στην έξοδο του \projectname{}.
Για παράδειγμα, οι λογικές διακλαδώσεις \engquote{If-else} παραθέτονται σειριακά στην έξοδο του γράφου του τρέχοντος συστήματος, σαν να εξαρτάται η μια από την άλλη.
Το υποτιθέμενο σύστημα για την παραγωγή κώδικα θα πρέπει να τις αναδιατάξει.

Συνοπτικά, το \projectname{} παρουσιάζει:
\begin{itemize}
    \item Μια προσέγγιση διαχωρισμού του προβλήματος σε μονάδες που επιλύουν ξεχωριστά υποπροβλήματα και συνδέονται μέσω μιας διασωλήνωσης λογισμικού.
          Η κάθε μονάδα υλοποιείται με τη χρήση βιβλιοθηκών ελεύθερου λογισμικού, στοχευμένων τόσο για ακαδημαϊκή όσο και για επαγγελματική χρήση, που χρησιμοποιούν σύγχρονες μεθόδους και προσφέρουν εκπαιδευμένα μοντέλα μηχανικής μάθησης.
          Το πλεονέκτημα αυτής της προσέγγισης είναι ότι με τη βελτίωση της απόδοσης αυτών των μοντέλων στο μέλλον το \projectname{} θα παρουσιάσει ανάλογες βελτιώσεις στα αποτελέσματά του.
          Για τις περισσότερες από αυτές τις αλλαγές δεν θα χρειάζονται ραγδαίες τροποποιήσεις στον κώδικα καθώς συνήθως οι βιβλιοθήκες διατηρούν συμβατότητα μεταξύ των διάφορων εκδόσεών τους.

          Επιπλέον, το καθένα από αυτά τα υποπροβλήματα αποτελεί σημαντικό πεδίο έρευνας και η βελτίωση των μοντέλων που το επιλύουν είναι συλλογική διαδικασία.
          Αυτή περιλαμβάνει την ενίσχυση των δεδομένων εκπαίδευσης και τη βελτίωση των αλγορίθμων μηχανικής μάθησης που εφαρμόζονται.
          Αντίθετα, ένα άκρη-προς-άκρη (\en{end-to-end}) σύστημα θα απαιτούσε τη συλλογή μεγάλου πλήθους δεδομένων εξειδικευμένων στο συγκεκριμένο πρόβλημα της διπλωματικής.

          Επίσης, αυτή η προσέγγιση επιτρέπει τον ευκολότερο εντοπισμό των πηγών των λανθασμένων αποτελεσμάτων όπως φαίνεται και στην \fref{sec:problems}.
    \item Τον βαθμό αποτελεσματικότητας μεθόδων μηχανικής μάθησης εκπαιδευμένων σε σχετικά μικρό αριθμό δεδομένων.
          Η δημιουργία του συνόλου εκπαίδευσης πραγματοποιήθηκε με περιορισμένους πόρους και εμπειρία στη συλλογή δεδομένων.
    \item Έναν καινοτόμο τρόπο διαχωρισμού πολλαπλών προθέσεων ανά πρόταση.
          Υποστηρίζονται δύσκολες δομές όπως αυτή της ανύψωσης δεξιού κόμβου.
          Δεν απαιτείται η συλλογή δεδομένων με παραδείγματα προτάσεων πολλαπλών προθέσεων.
          Αυτά θα πλήθυναν σημαντικά τον απαιτούμενο αριθμό δεδομένων και θα απαιτούσαν μια πολυωνυμική αύξησή τους με κάθε νέα πρόθεση στο σύστημα.
    \item Ένα σύστημα που μπορεί να αποδειχθεί χρήσιμο εργαλείο σε έναν χρήστη εφόσον μάθει τους περιορισμούς του και τον τρόπο σύνταξης που οδηγεί σε βέλτιστα αποτελέσματα.
          Αυτή η διαδικασία μάθησης θεωρείται ότι δεν απαιτεί τεχνικές και εσωτερικές γνώσεις πάνω στο \projectname{}.
\end{itemize}

Περιορισμοί του συστήματος και πιθανά ελαττώματα είναι:
\begin{itemize}
    \item Η τμηματοποίηση σε μονάδες μπορεί να οδηγήσει σε απώλεια πληροφορίας.
          Δηλαδή, η εκπαίδευση των μοντέλων ενός υποπροβλήματος μπορεί να επωφεληθεί από τις πληροφορίες που συνδέονται με κάποιο άλλο υποπρόβλημα λόγω της μεταξύ τους στατιστικής συσχέτισης.
          Για παράδειγμα, ένα κοινό μοντέλο για τον διαχωρισμό και αναγνώριση προθέσεων μπορεί να λάμβανε υπόψη λεπτομέρειες σχετικές με τον συγκεκριμένο συνδυασμό δύο προθέσεων σε μια πρόταση.
    \item Τα μοντέλα που επιλύουν γενικά προβλήματα, αν και παρουσιάζουν καλύτερες επιδόσεις στη γενική περίπτωση, μπορεί να εμφανίζουν ελλείψεις λόγω διαφοράς λεξιλογίου στον συγκεκριμένο τομέα του προβλήματος.
    \item Δεν παρουσιάζει καλή απόδοση σε κείμενα που απαιτούν υψηλότερα επίπεδα λογικής όπως για παράδειγμα σε προτάσεις σαν αυτές που εμφανίζονται σε δοκιμασίες \lib{Winograd}~\cite{levesque2012winograd}.
    \item Απαιτείται η είσοδος γραμματικά, συντακτικά και ορθογραφικά ορθού κειμένου με τις παραθέσεις να περιλαμβάνονται πάντα μέσα σε εισαγωγικά.
          Αυτή η έλλειψη ευρωστίας μπορεί να δυσχεραίνει τη σύνδεση του \projectname{} με την έξοδο με ένα σύστημα \newterm{Αυτόματη\dd{ς} Αναγνώριση\dd{ς} Ομιλίας}{Automatic Speech recognition}.
    \item Κατά την αναγνώριση πρόθεσης δεν γίνεται εκμετάλλευση των συμφραζομένων των προηγούμενων προτάσεων.
          Ο μόνος τρόπος που οι περασμένες προτάσεις επηρεάζουν την επεξεργασία της τρέχουσας πρότασης είναι μέσω συναναφορών.
\end{itemize}

\section{Μελλοντική εργασία}
Είναι δυνατή η αντιμετώπιση μερικών περιορισμών του συστήματος με μελλοντικές επεκτάσεις.
Κάποιες βασικές ιδέες είναι:
\begin{itemize}
    \item Χρήση ρομποτικών οντολογιών.
          Οι \newterm{Οντολογί\rr{ες}{α}}{Ontology} παρέχουν έναν δομημένο τρόπο οργάνωσης πληροφοριών,
          υποστηρίζουν μεθόδους για την ανάκτηση αποθηκευμένων δεδομένων μέσω \newtermprint[Query]{ερωτημάτων}
          και επιτρέπουν την ανάλυση των σχέσεων μεταξύ δεδομένων~\cite{diamantopoulos2017software}.

          Επιτρέπουν την ενσωμάτωση πληροφοριών του εξωτερικού κόσμου και μπορούν να βοηθήσουν στην καλύτερη κατανόηση των προθέσεων του χρήστη.
          Για παράδειγμα, οι \citet{hu2009understanding} χρησιμοποιούν τη \en{Wikipedia} ως οντολογία σε συνδυασμό με τυχαίους περιπάτους για την αναγνώριση της πρόθεσης του χρήστη.
          Στο~\cite{zang2018translating} οι εντολές του χρήστη συνδέονται με έναν χάρτη του περιβάλλοντος.
          Στα~\cite{salamknowledge,zhang2018learning,nyga2018grounding} οι οντολογίες χρησιμοποιούνται έτσι ώστε το ρομπότ να έχει καλύτερη κατανόηση των οντοτήτων που περιέχονται στα λεγόμενα του χρήστη.
          Τέλος, οι \citet{diamantopoulos2017software} χρησιμοποιούν οντολογία για την ανάλυση λειτουργικών απαιτήσεων.

          Η υλοποίηση μιας οντολογίας που μπορεί να καλύψει επαρκώς τη χρήση του \metamodel{} θεωρήθηκε εκτός του πλαισίου αυτής της διπλωματικής.
          Επίσης, μια οντολογία που συνδέεται με το τρέχον περιβάλλον του ρομπότ (όπως για παράδειγμα το \libcite{KnowRob}) προσφέρει πληροφορίες που δεν γενικεύονται για κάθε ρομποτική εφαρμογή.
    \item Βελτίωση της ανίχνευσης φυσικών μονάδων απόστασης και ταχύτητας που χρησιμοποιούνται για την πρόθεση \intent{BodyMotion}.
          Αυτό μπορεί να γίνει με υλοποιήσεις επεκτάσεων στα \libcite[Python!]{Duckling} ή \libcite{Rustling} και την καλύτερη ενσωμάτωσή τους στην τρέχουσα σωλήνωση λογισμικού.
    \item Αναγνώριση προθέσεων που λαμβάνει υπόψη το ιστορικό των προηγούμενων προτάσεων.
          Ιδέες μπορούν να αντληθούν από τα συστήματα διαλόγων~\cite{bhargava2013easy} ή από άλλες δημοσιεύσεις ταξινόμησης κειμένου~\cite{lee2016sequential}.
          Αυτός ο στόχος είναι δύσκολο να εκπληρωθεί στο πλαίσιο αυτής της διπλωματικής λόγω της ανάγκης ύπαρξης μεγάλου αριθμού δεδομένων με ολοκληρωμένα σενάρια χρήσης όπως αυτά που εμφανίζονται στο \fref{chap:results}.

          Επιπρόσθετα, ίσως είναι εφικτή και η ενίσχυση του μοντέλου αναγνώρισης πρόθεσης με παρόμοια χαρακτηριστικά.
    \item Εκμετάλλευση του αποτελέσματος της ανάθεσης σημασιολογικών ρόλων για τη βελτίωση της απόδοσης πλήρωσης υποδοχών στη μονάδα \NLU{}.
          Μπορούν να δημιουργηθούν συναρτήσεις χαρακτηριστικών για τα \CRFR{} που το αποτέλεσμά τους να εξαρτάται από τον σημασιολογικό ρόλο κάθε λέξης.
    \item Αυτόματη αναγνώριση παραθέσεων που δε βρίσκονται ανάμεσα σε εισαγωγικά και πλάγιου λόγου όπως, για παράδειγμα, στο~\cite{pareti2013automatically}.
    \item Επανεκπαίδευση του \libcite[Python!]{spaCy} σε προτάσεις στις οποίες παρουσιάζει πρόβλημα στην επισημείωση μερών του λόγου (βλέπε και \fref{subsec:problems-spacy}).
          Αυτό θα βοηθήσει στην καλύτερη απόδοση του μοντέλου στο συγκεκριμένο λεξιλόγιο που συναντάται στα σενάρια χρήσης του \projectname{}.
    \item Αξιοποίηση μεταφοράς μάθησης μέσω \libcite{BERT}.
          Αυτό μπορεί να οδηγήσει σε βελτίωση της απόδοσης των μοντέλων του \libcite[Python!]{AllenNLP}\footnote{\url{https://github.com/allenai/allennlp/pull/2854}}.
          Μπορεί να διερευνηθεί και η αξιοποίησή τους στο πρόβλημα της επίλυσης σχέσεων συναναφοράς.

          Επίσης, είναι πιθανή η εξαγωγή \newterm{\rr{Διανυσμάτων}{Διανύσματα} Λέξεων}{Word Vectors} και η σύγκρισή τους με αυτά των \libcite{GloVe} και \libcite{ELMO}.

          Στη παρούσα διπλωματική, διερευνήθηκε η χρήση ταξινόμησης μέσω νευρωνικών δικτύων βασισμένων στο \lib{BERT} αλλά τελικά απορρίφθηκε καθώς αποτελεί υπολογιστικά χρονοβόρα διαδικασία,
          ενώ η τρέχουσα μονάδα \NLU{} βασίζεται στην αξιολόγηση πολλών συνδυασμών ανά πρόταση.
          Για να προσπεραστεί αυτό το πρόβλημα, είναι δυνατό να επιχειρηθεί η εκπαίδευση μοντέλου ταξινόμησης ακολουθιών πολλών ετικετών βασισμένου και πάλι στο \lib{BERT} που θα αντικαταστήσει ολικά τη μονάδα NLU.
          Μια άλλη πιθανή προσέγγιση εδώ είναι της \newterm{Μάθησης Πολλών Στόχων}{Multi-Task Learning} κατά την οποία συνήθως επιχειρείται η ταυτόχρονη επίλυση πολλών προβλημάτων,
          δηλαδή η βελτιστοποίηση γίνεται πάνω σε πολλές συναρτήσεις απωλειών.
\end{itemize}

% vim:ts=4:sw=4:expandtab:fo-=tc:tw=120


\clearpage{}
\printindex{}

\bibliography{%
    bibliography/general%
    ,bibliography/background%
    ,bibliography/coref%
    ,bibliography/dialogue-systems%
    ,bibliography/srl%
    ,bibliography/semantic%
    ,bibliography/libraries%
    ,bibliography/transfer-learning%
    ,bibliography/future%
}

\end{document}

% vim:ts=4:sw=4:expandtab:fo-=tc:tw=120
