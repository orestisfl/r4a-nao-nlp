\section{Επίλυση σχέσεων συναναφοράς}
Η επίλυση σχέσεων συναναφοράς (βλέπε και \fref{subsec:coref}) εξακολουθεί να θεωρείται ένα ανοιχτό πρόβλημα λόγω της περίπλοκης και διφορούμενης φύσης της φυσικής γλώσσας.
Οι υπάρχουσες μέθοδοι συναντούν ιδιαίτερα προβλήματα σε αντωνυμίες, ειδικά όταν δεν μπορούν να εξαχθούν πληροφορίες σχετικά με το φύλο ή τον αριθμό της οντότητας που εξετάζεται~\cite{peng2015solving} αλλά πρέπει να χρησιμοποιηθεί γνωστικό υπόβαθρο που είναι δύσκολο να αποκτηθεί από μια μηχανή~\cite{ng2017machine}.
Σε αυτή την κατεύθυνση, οι \citet{rahman2011coreference} και \citet{zhang2019incorporating} προσπαθούν να ενσωματώσουν στα μοντέλα τους πληροφορίες του εξωτερικού κόσμου για να ενισχύσουν την απόδοσή τους.

Πρόσφατες υλοποιήσεις συστημάτων επίλυσης συναναφορών αξιοποιούν αρχιτεκτονικές νευρωνικών δικτύων.
Το~\cite{CorefAnnotatorN1} δημιουργεί \newtermprint[Vector Representations]{Διανυσματικές Αναπαραστάσεις} υψηλών διαστάσεων για κάθε ζεύγος συναναφορικών ομάδων.
Το σύστημα μαθαίνει πώς να συγχωνεύει τις ομάδες αυτές για την παραγωγή καλύτερου αποτελέσματος.
Επίσης, αναφέρεται και το μοντέλο \newtermprint[Mention Ranking]{κατάταξης αναφορών} όπου αξιολογεί την πιθανότητα δύο ονοματικών φράσεων να ανήκουν στην ίδια συναναφορική ομάδα.
Στο~\cite{CorefAnnotatorN2} το ίδιο μοντέλο κατάταξης αναφορών βελτιστοποιείται με τη χρήση ενισχυτικής μάθησης.
Τέλος, οι \citet{lee2017end} δημιούργησαν ένα άκρη-προς-άκρη μοντέλο για την επίλυση συναναφορών και αναφέρουν, αυτή τη στιγμή, τα καλύτερα αποτελέσματα στον κλάδο.
Το σύστημά τους δεν χρησιμοποιεί συντακτική ανάλυση ή ανιχνευτές αναφορών, κύριο συστατικό του μοντέλου αποτελούν διανύσματα λέξεων που αναπαριστούν διαστήματα του κειμένου.

Στον χώρο του ελεύθερου λογισμικού, η βιβλιοθήκη \libcite[Python!]{neuralcoref} αξιοποιεί ιδέες από τα~\cite{CorefAnnotatorN1,CorefAnnotatorN2}.
Αυτές οι δημοσιεύσεις χρησιμοποιούνται και στο νευρωνικό μοντέλο της εργαλειοθήκης \libcite{CoreNLP},
ωστόσο η \lib{neuralcoref} πραγματοποιεί την υλοποίηση ως επέκταση της πιο ελαφριάς βιβλιοθήκης \libcite{spaCy}.
Επιπλέον, αξιοποιεί μια πιο πρόσφατη δημοσίευση για τον υπολογισμό διανυσμάτων αγνώστων λέξεων~\cite{bahdanau2017learning}
και λαμβάνει υπόψη τον ομιλητή~\cite{raghunathan2010multi} ---
για παράδειγμα, στη πρόταση \engquote{\engquote{I voted my conscience}, she said} οι λέξεις \engquote{I} και \engquote{she} αναφέρονται στο ίδιο πρόσωπο.
Η \libcite{AllenNLP} επίσης χρησιμοποιεί το άκρη-προς-άκρη μοντέλο νευρωνικών δικτύων που αναφέρθηκε προηγουμένως~\cite{lee2017end} το οποίο μπορεί να προσφέρει καλύτερα αποτελέσματα μέσω του \libcite{ELMO}.

% vim:ts=4:sw=4:expandtab:fo-=tc:tw=120
