\section{Ανάθεση σημασιολογικών ρόλων}
Η δημιουργία μοντέλων που προσπαθούν να λύσουν το πρόβλημα της ανάθεσης σημασιολογικών ρόλων (\SRLR{}) συνήθως βασίζεται σε ένα σύνολο δεδομένων που περιέχει μεγάλες συλλογές κειμένων για τις οποίες οι σημασιολογικοί ρόλοι έχουν ανατεθεί χειροκίνητα από ανθρώπους.
Οι δύο πιο συχνά χρησιμοποιούμενες είναι η \libcite{FrameNet} και η \libcite{PropBank}.

Παραδοσιακά, μοντέλα που επιχειρούσαν να επιλύσουν το πρόβλημα θεωρούσαν απαραίτητη~\cite{carreras2005introduction,punyakanok2008importance} τη \newtermprint[Syntactic Parsing]{Συντακτική Ανάλυση} της κάθε πρότασης δηλαδή,
την αντιστοίχηση μιας συντακτικής δομής σε κάθε πρόταση.
Ένα παράδειγμα τέτοιας υλοποίησης γίνεται από τους \citet{koomen2005generalized} στο οποίο αξιοποιείται μαθηματική βελτιστοποίηση με περιορισμούς σε συνδυασμό με μοντέλα ταξινόμησης.

Ωστόσο, η πρόσφατη αποτελεσματικότητα των βαθιών νευρωνικών δικτύων οδήγησε στην υλοποίηση άκρη-προς-άκρη (\en{end-to-end}) μοντέλων.
Οι \citet{collobert2011natural} ήταν οι πρώτοι που εφάρμοσαν CNN για την επίλυση αυτού του προβλήματος.
Έπειτα, εμφανίστηκαν αρκετές υλοποιήσεις~\cite{zhou2015end,marcheggiani2017simple,he2017deep} που αξιοποιούν την καλή απόδοση των LSTM δικτύων σε ακολουθιακά δεδομένα.

Η βιβλιοθήκη κατανόησης φυσικής γλώσσας \libcite[Python!]{AllenNLP} χρησιμοποιεί το μοντέλο από το~\cite{he2017deep} που παρουσιάζει αυτή τη στιγμή τα καλύτερα αποτελέσματα που αναφέρονται στη βιβλιογραφία.
Για την ενίσχυση της απόδοσης χρησιμοποιήθηκαν τα διανύσματα λέξεων \libcite{ELMO} αντί για αυτά του \libcite{GloVe} της αρχικής δημοσίευσης.

% vim:ts=4:sw=4:expandtab:fo-=tc:tw=120
